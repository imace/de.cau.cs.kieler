\chapter{Conclusion}
\label{chapter:AutoConclusion}
As stated in Chapter \ref{chapter:AutoTask} the problem consists of four parts:
\begin{enumerate}
 \item Find an easy way for the user to set up an automated run.
 \item Input the information provided by the user into the data components.
 \item Design a control flow for an automated run.
 \item Organize the output of the automated run and display it to the user.
\end{enumerate}
The following sections will summarize the results and provide some ideas for
future work.

\section{Results}
\label{section:AutoResults}
The first objective of the project was to solve the set up of an automated run
by the user. The initial idea of using a script-based approach and providing the
necessary files as lists in text files was not pursued. Instead a more user-friendly
solution was found through the use of a wizard. 

The next objective was to find a way to input information into the DataComponents
in order for them to be designed in a more generic way. This was achieved by
designing new interfaces that allow components to receive a list of properties
prior to each part of the automated execution.

The main part of the task involved creating a manager that guided the control flow
of an automated execution. The Automation Manager described in Section \ref{section:AutomationManager}
loads all the necessary files, sets up the different outputs, steps through the
execution to the desired step and includes error management facilities as well.

The last part was to find a user-friendly way to display the information generated
by the automated execution. The problem was solved by creating a view that displayed
the information in a set of tables. These tables are structured in a way to easily
compare the results of different model files and iterations. To allow use of the 
generated tables in another context methods were implemented in order to allow
the generation of external formats (namely \ac{CSV} and \LaTeX).


\section{Future Work}
\label{section:AutoImprovements}
Although all initial objectives were achieved there is still room for additional
improvements. These improvements which could be the subject of further study will
be explained in this section.

\subsection{Scripting}
Currently the automated execution can be triggered through
the use of a wizard and will run inside the Eclipse workbench and display the results
in a graphical view. 

In order to allow other plug-ins or external applications to trigger
an automated execution an interface would have to be defined. This could involve the 
creation of a scripting language that passes all the needed parameters to the automated
execution. It would also involve retrieving those results and importing them into the 
calling application.

The existing code already supports most of the requested features however there is no
interface to access those features from outside of Java.

\subsection{Exports}
Currently the only possibility to use the collected data in
another application is by exporting the entire table into \ac{CSV} or \LaTeX .

This process could be improved, for example by allowing the user to select the table columns
and rows that he wants to export instead of exporting the entire table. Furthermore
export to additional formats could be implemented. It could even be possible to interface
the entire automation with a database application or remote system.

\section{Summary}
All in all the initial problem of providing a framework for setting up automated execution runs
was solved. 

Further testing and use of these features may however reveal new ways that the
plug-in can extend its functionality.
