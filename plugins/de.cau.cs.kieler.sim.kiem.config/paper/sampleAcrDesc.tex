\documentclass[a4paper]{report}

\usepackage[colorlinks,plainpages=false]{hyperref}

\usepackage[style=altlist, % use altlist style
            toc, % add the glossary to the table of contents
            sanitize={description=false},% want to use description in main document
            description% acronyms have a user-supplied description
           ]{glossaries}

\makeglossaries

\newacronym[description={Statistical pattern recognition
technique~\protect\cite{svm}}, % acronym's description
name={Support vector machine (svm)}% change the default way of displaying the entry name in the list of acronyms
]{svm}{svm}{support vector machine}

\newacronym[description={Statistical pattern recognition technique
using the ``kernel trick'' (see also \glshyperlink[SVM]{svm})},% acronym's description
name={Kernel support vector machine (ksvm)}% change the default way of displaying the entry name in the list of acronyms
]{ksvm}{ksvm}{kernel
support vector machine}

\begin{document}
\tableofcontents

\chapter{Support Vector Machines}

The \gls{svm} is used widely in the area of pattern recognition.
 % plural form with initial letter in uppercase:
\Glspl{svm} are \ldots

This is the text produced without a link: \glsentrytext{svm}.
This is the text produced on first use without a link:
\glsentryfirst{svm}. This is the entry's description without
a link: \glsentrydesc{svm}.

This is the entry in uppercase: \GLS{svm}.

\chapter{Kernel Support Vector Machines}

The \gls{ksvm} is \ifglsused{svm}{an}{a} \gls{svm} that uses
the so called ``kernel trick''. This is the entry's description without
a link: \glsentrydesc{ksvm}.

\glsresetall
Possessive: \gls{ksvm}['s].
Make the glossary entry number bold for this
one \gls[format=hyperbf]{svm}.

\begin{thebibliography}{1}
\bibitem{svm} \ldots
\end{thebibliography}

\printglossary

\end{document}
