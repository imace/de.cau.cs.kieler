
\chapter{C++-Code f�r die GUI}
\label{sec:code}

\section{Beschreibung}



\begin{description}

\item[\texttt{haupt.cpp}:]
In der Datei haupt.cpp ist die Klasse \code{Nebenbei}, sowie die Instantiierung
des \eng{Widgets} beschrieben. Die Klasse \code{Nebenbei} realisiert die
Nebenl�ufigkeit des Programms. Sie besteht aus einem Konstruktor, der
bekanntlich genauso hei�t wie seine Klasse und der Methode \code{run}. In 
der Methode \code{run} wird der parallel zum \gui\ laufende Code
definiert. Hier wird als erstes die Digitaluhr instantiiert. In der
\code{while}-Schleife werden die \eng{Automation Engine}- und
\eng{Input}-Funktionen aufgerufen. Mit der Abfrage, ob der Wert der zum
\eng{Input}-Signal passenden Variable gleich zwei ist wird �berpr�ft, ob der
Benutzer dieses Signal emittieren m�chte. 

\item[\texttt{Gui.h}:]
In dieser \eng{Header}-Datei sind die Klasse \code{GraficWidget}, ihre
\eng{Slots} und ihre Methoden definiert. Au�erdem sind hier die globalen
Variablen definiert, die die Kommunikation zwischen \eng{Thread} und
\gui\ realisieren. 

\end{description}


\section{Der Code}

\begin{landscape}
  \lstfile{code/}{haupt.cpp}
  \lstfile{code/}{Gui.cpp}
\end{landscape}


%% Alternativ:

%% \subsection{\texttt{haupt.cpp}}

%% \VerbatimInput[%
%%   frame=topline,
%%   framesep=2mm,
%%   label=haupt.cpp,
%%   fontsize=\footnotesize,
%%   numbers=left]%
%%   {code/haupt.cpp}

%% \newpage

%% \subsection{\texttt{Gui.h}}

%% \VerbatimInput[%
%%   frame=topline,
%%   framesep=2mm,
%%   label=GUI.h,
%%   fontsize=\footnotesize,
%%   numbers=left]%
%%   {code/Gui.h}





%%% Local Variables: 
%%% mode: latex
%%% TeX-master: "paper"
%%% End: 
