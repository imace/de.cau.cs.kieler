\documentclass[11pt,a4paper,BCOR10mm,bibtotocnumbered]{scrbook}

%% \usepackage{abstract}          % typesetting of a abstract environment

\usepackage{acronym}          % acronyms
\usepackage{amscd}            % Kommutative Diagramme
\usepackage{amsmath}          % mathematics
\usepackage{amsfonts}         % mathematics
\usepackage{array}            % tabels
\usepackage{fancyvrb}         % Umgebung f�r Programm-Listings
\usepackage{float}            % mehr Positionen von Float-Umgebungen
\usepackage[T1]{fontenc}      % Fonteinstellung
\usepackage{graphicx}         % Laden von Bildern
%\usepackage{glossary}         % Erstellen eines Glossars DONT EVER USE THIS
\usepackage[nonumberlist]{glossaries}
\usepackage{hyperref}         % Querreferenzen und Links
\usepackage[latin1]{inputenc} % Zeichenkodierung
\usepackage{keystroke}        % Tastendarstellung
\usepackage{lscape}           % Landscape 
\usepackage{makeidx}          % Indexerstellung
\usepackage{mflogo}           % LaTeX-Logos
\usepackage{multicol}         % mehrspaltiger Satz
\usepackage{multirow}
\usepackage[numbers]{natbib}  % Autor-Datum Zitierungen
%\usepackage{ngerman}          % Zur Verwendung der deutschen Sprache
\usepackage{rotating}         % Rotieren von Bildern und Texten
\usepackage{syntax}           % Syntax-Diagramme
\usepackage{typearea}         % Satzspiegelberechnung
\usepackage{url}              % Hyperlinks
\usepackage{xspace}           % Korrekter Leerraum nach Befehlen
\usepackage{listings}         % Auflistung von Ausschnitten von Programm-Code
\usepackage[T1]{fontenc}      % ben�tigt von luximono
\usepackage{subfigure}        % Unterabbildungen
\usepackage{textcomp}         % Verschiedene Zeichen im textmode
\usepackage{xcolor}
\usepackage{courier}
 


\usepackage[bachelorproject,english]{rtsstud}          % Lehrstuhl-Spezifikationen
\usepackage{rtsabstr}         % Zusammenfassung

\usepackage{mystyle}          % eigene LaTeX-Spezifikationen
  
\newcommand{\listingxml}{
\lstset{
basicstyle=\scriptsize\tt,
numbers=left,
frame=shadowbox,
tab=.,
language=XML,
tabsize=2,
showspaces=false,
showstringspaces=false,
captionpos=b,
basicstyle=\footnotesize,
keywordstyle=\tiny\bfseries,
numbers=left,
numberstyle=\tiny,
numberblanklines=false,
}}
\newcommand{\listingjava}{
\lstset{
basicstyle=\scriptsize\tt,
numbers=left,
frame=shadowbox,
tab=.,
rulesepcolor=\color{lightgray},
morecomment=[l]{//},
tabsize=3,
language=Java,
keywordstyle=\bfseries\color[rgb]{0.482,0,0.323},
stringstyle=\color{blue},
commentstyle=\color[rgb]{0.224,0.49,0.353},
numberstyle=\tiny,
breaklines=true,
showstringspaces=false,
showtabs=false,
showspaces=false,
backgroundcolor=\color[rgb]{1,1,0.74},
emph={},
emphstyle=\bfseries\color[rgb]{0.482,0,0.323}
% frame=single, %single
% backgroundcolor=\color[rgb]{1,1,0.6}
}}

\newcommand{\showlistingex}[5]{
\begin{figure}[#5]
{
% \hspace{0.4cm}
\scriptsize
\lstinputlisting[language=#2,label=#4,caption=#3]{#1}
\normalsize
% \hspace{0.1cm}
}
\end{figure}
}



\title{Configurations and Automated Execution in the KIELER Execution Manager}
\subtitle{~}
\author{cand.~inform.~S\"oren Hansen}
\date{\today}
\advisor{%
  \renewcommand{\thefootnote}{\fnsymbol{footnote}}%
  Christian Motika\\
}

% {%
% \renewcommand{\thefootnote}{\fnsymbol{footnote}}
% \protect\footnotemark
% \footnotetext[1]{Das ist eine Anmerkung zur Person.}
% }

\makeindex


\begin{document}

\frontmatter
\maketitle
% \assertion
% \include{abstract}

\makeatletter
\@ifundefined{float@listhead}{}{%
    \renewcommand*{\lstlistoflistings}{%
        \begingroup
    	    \if@twocolumn
                \@restonecoltrue\onecolumn
            \else
                \@restonecolfalse
            \fi
            \float@listhead{\lstlistlistingname}%
            \setlength{\parskip}{\z@}%
            \setlength{\parindent}{\z@}%
            \setlength{\parfillskip}{\z@ \@plus 1fil}%
            \@starttoc{lol}%
            \if@restonecol\twocolumn\fi
        \endgroup
    }%
}
\makeatother


%% Inhaltsverzeichnis
\tableofcontents

%% Inhaltsverzeichnis
%\listoftables

%% Inhaltsverzeichnis
\listoffigures

%% Verzeichnis der Auflistungen
\lstlistoflistings

%% Verzeichnis der Abk�rzungen

%% Abbreviations
\chapter*{Abbreviations}
\begin{acronym}
 \acro{API}{Application Programming Interface}
 \acro{GUI}{Graphical User Interface}
 \acro{IDE}{Integrated Development Environment}

 \acro{KIEL}{Kiel Integrated Environment for Layout}
 \acro{KIELER}{Kiel Integrated Environment for Layout Eclipse Rich Client}
 \acro{KIEM}{KIELER Execution Manager}
 \acro{KIEMAuto}{Automated Executions for the KIEM}
 \acro{KIEMConfig}{Configurations for the KIEM}

 \acro{OS}Operating System
 \acro{RCA}Rich-Client Application
 \acro{UI}User Interface
\end{acronym}


%%% Local Variables: 
%%% mode: latex
%%% TeX-master: "paper"
%%% End: 


%% Verzeichnis der Akronyme
% \input{acronyms}

%%% Local Variables: 
%%% mode: latex
%%% TeX-master: "paper"
%%% End: 
% \include{acronyms}

\mainmatter
\chapter{Introduction}
\label{chapter:introduction}
The purpose of this thesis consists of two parts. The first part is to find an easier way and
more flexible way to deal with execution files in the \ac{KIELER} Execution Manager. The second part
is to find an easy way to automatically do long execution runs inside the \ac{KIEM}.

\section{KIELER Framework}
\label{sec:introKieler}
\index{KIELER}
Since the project is part of the \ac{KIELER}\footnote{\url{http://www.informatik.uni-kiel.de/rtsys/kieler/}}
framework a short introduction is in order.

\ac{KIELER} is an open-source project for model design, simulation and analysis. 

It contains facilities to edit different types of graphical models (e.g. synccharts) and has several
tools to make the editing easier. The \ac{KSBasE}\footnote{\url{http://rtsys.informatik.uni-kiel.de/trac/kieler/wiki/Projects/KSBasE}} 
project for example contains several features where the user can edit successor states to already existing states in the diagram or
adding new inner states without having to perform all the necessary actions himself. Another 
project\footnote{\ac{KIML} - \url{http://rtsys.informatik.uni-kiel.de/trac/kieler/wiki/Projects/KIML}}
automatically layouts any diagram with a variety of algorithms in order to improve readability.

The \ac{KIELER} Execution Manager described in the next section is used for simulating the
model files created by \ac{KIELER}.

\subsection{Model Files}
\label{section:IntroModelFile}
\index{Model file}
A model file can be any file that contains the structure and behavior of a semantic entity.

\section{The KIELER Execution Manager}
\label{sec:introKiem}
\index{KIEM}
Execution Manager is used in KIELER as a framework to plug-in DataComponents for various tasks. Examples are:
\begin{itemize}
 \item Simulation Engines
 \item Model Visualizers
 \item Environment Visualizers
 \item Validators
 \item User Input Facilities
 \item Trace Recording Facilities 
\end{itemize}

These DataComponents can be executed using a graphical user interface (GUI). 
The scheduling order can also be defined by this GUI as well as other settings like a step/tick duration and properties of DataComponents.
For information about \ac{KIEM} see the wiki\footnote{\url{http://rtsys.informatik.uni-kiel.de/trac/kieler/wiki/Projects/KIEM}}.

\subsection{DataComponents}
\label{section:IntroDataComponent}
\index{DataComponent}
A DataComponent is an entity that has a specific task during a the course of an execution. The
DataComponents are scheduled in a specific order and can either receive or produce information or both. 
During each step of the execution each DataComponent is asked to perform their computations for the step.

\subsubsection{DataComponentWrappers}
\label{section:IntroDataComponentWrapper}
\index{DataComponentWrapper}
A DataComponentWrapper is an object that wrapps around a DataComponent that is contained in the schedule
of an execution file.


\subsection{Execution}
\label{section:IntroExecution}
\index{Execution file}
\index{Execution}
\index{Schedule}
The files used by the Execution Manager are called \textit{execution files}. These files contain 
the list of DataComponents with all internal data and the specific order. That list is called a
\textit{schedule}. The schedule can be used to start an \textit{execution} in the Execution Manager.
An execution consists of a initialization, stepping and wrap-up.


\section{Outline of this Document}
\label{sec:introOutline}
The first part of this document is about the implemention of the Configuration plugin for the \ac{KIEM}.
The second part discusses the implementation of the Automated Execution plugin for the \ac{KIEM}. Both
parts will have the same structure described below.

Each part starts with a brief introduction into the technologies used to solve the problem.

The next chapter contains a detailed outline of the problems that are to be solved by this thesis.

The following chapter describes the concepts of how to solve the problem and some
design decisions that were made at a very early stage in the development.

Sections 5 and 6 are about the actual implementation with section 5 outlining the changes
that were made to the \ac{KIEM} plug-in itself and section 6 describing the newly created plug-ins.

The final section of each part will summarize the results of the thesis and outline
a few projects that could be based on it.


%%% Local Variables: 
%%% mode: latex
%%% TeX-master: "paper"
%%% End: 



\part{Configuration Management}
\chapter{Used technologies}
\label{chapter:ConfTechnology}
- before explaining solutions

- short explanation about the technology in question

- small sketch to get an idea of the context

- full explanation goes beyond the scope of this work

\section{Eclipse}
\index{Eclipse}
- KIELER part of Eclipse framework

- Java IDE but also lots of other languages (C++, Latex, ...)

- can build other IDEs with it

- citation: IDE an for anything, and nothing in particular \cite{eclipseOverview}

\subsection{Plug-ins}
\index{Plug-in}
- \ac{KIEM} itself is a plug-in, this work plug-in dependent on \ac{KIEM}

- different components of an \ac{IDE} 

- can operate by themselves or depending on other plug-ins

- in addition to \ac{API} of each plug-in: extension point mechanism

- extension points to interact with other plug-ins

- extension point mechanism used by runtime environment to check which plug-ins to load

- plug-ins can be used to plug into the standard Eclipse feature
\subsubsection{Extension point mechanism}
\label{section:TechExtension}
\index{Extension point}


\subsection{Preference Pages}
\label{section:TechPreferencePage}
\index{Preference Page}
- One of the plug-ins used in this work in that fashion is the \emph{org.eclipse.ui.preferencePage}
plug-in. It is used to create new preference pages at a specific location inside the
normal tree of preference pages accessible through Window>Preferences.
The programmer only has to take care of the contents of the actual page and not worry
about additional buttons or integrating it into the PreferenceDialog.

\subsubsection{Preference Store}
\label{section:TechPreferenceStore}
- Closely coupled with the preference pages is the Eclipse preference store. It is
basically a text file for each plug-in where the plug-in can deposit simple Strings
under a given key to ensure that information is kept between execution runs.

For additional information about eclipse see the official Eclipse website\footnote{www.eclipse.org} or literature \cite{eclipsePlugins}
\chapter{Problem Statement}
\label{chapter:ConfTask}
The objective of this project is to improve the configurability of the
Execution Manager as outlined in the diploma-thesis by Christian Motika\cite{cmot-dt} :
\begin{quote}
 \ac{KIEM} currently does not have a preference page to save
additional settings like DataComponent timeouts. Also execution schedulings
might be similar for a common diagram type.

It may improve the usability further to allow the user to customize execution
schedulings for specific diagram types. An interface for these kind of settings
could be realized as an Eclipse preference page.
\end{quote}

This task will be explained in more detail in this chapter. It will start by introducing 
solutions to the problem of how save the new configuration properties into the existing 
execution files. In addition the chapter will explain ways to enable the user to set up a 
series of default configurations. The last section of this chapter will explore 
possibilities of how to make it easier to load previously saved schedules.

\section{Configurations}
\label{section:ConfTaskConfig}
Currently every property in the Execution Manager has a hard coded default value. There is a text box
for setting the aimed step duration for the currently loaded execution file but that value
is lost once a new execution file is loaded.
To solve this problem an extension to the Execution Manager should attempt to provide the following:
\begin{enumerate}
 \item Find a way that execution files can store values like the aimed step duration and the timeout.
This mechanism should be implemented in a way that ensures that old files can be upgraded and new
files are still valid in instances of the Execution Manager that don't use the configuration plug-in.
 \item Find a way to load the configurations into the different parts of the Execution Manager as soon as an
execution file is loaded from the file system.
 \item Ensure that the user can edit all properties and maybe even create his own custom properties.
This should be implemented in a way that doesn't clutter up the current user interface too much.
\end{enumerate}

\subsection{Default Configuration}
\label{section:ConfTaskDefaultConfig}
\index{Default Configuration}
The different properties stored in each execution file might sometimes not suit the users current needs
and he might want to use a default value for some properties without having to manually set them
in each new configuration.
The solution could be implemented using the preference mechanism provided by Eclipse.
\begin{enumerate}
 \item There should be a way to set the default properties for all Execution Manager properties 
and possibly for user defined properties as well.
 \item It should be possible for the user to set up which of these properties should override 
the value stored in the execution file and which
should only be used if the execution file doesn't contain one.
\end{enumerate}

\section{Easier Configuration Loading}
\label{section:ConfTaskEasyLoading}
The last objective of this thesis is to make it easier to load execution files.
Currently all execution files are stored in the workspace at a place of the users choice. In a very
large workspace it can be very hard to find the execution file that you need for your current
simulation. The list of recently used documents that Eclipse provides is of little use since all
opened documents are placed there not just execution files.
This problem leads to the in the following tasks:
\begin{enumerate}
 \item Finding a way to track recently used execution files and make it easier for the user to
load them without the need to locate them inside his workspace.
 \item In addition to tracking recently used execution files the user might want to have a way to get a list
of execution files that work for the currently active editor. This list should be sorted with the most likely
candidates at the top to allow less experienced users to select an execution file that will most likely work.
\end{enumerate}

\chapter{Concepts}
\label{chapter:ConfConcepts}
The solution to the problems outlined in Chapter \ref{chapter:ConfTask} can be achieved with
the help of the Eclipse plug-in technology described in Chapter \ref{chapter:ConfTechnology}.

This chapter explains the solutions in the same structure as Chapter \ref{chapter:ConfTask}.
That means that it will start by introducing solutions to the problem of how save the new
configuration properties into the existing execution files. In addition the chapter will
explain ways to enable the user to set up a series of default configurations. The last
section of this chapter will explore possibilities of how to make it easier to load
previously saved schedules.

\section{Configurations}
\label{section:ConfConceptsConf}
\index{Execution file}
\index{DataComponent}
\index{Configuration}
The first approach to save additional configuration information in the execution files (see Section \ref{section:IntroExecution}) would
be to actually modify the format of those files and write the information into them.
This would most likely be the easiest approach but would destroy backward compatibility of
those files since the basic Execution Manager would not know how to deal with the modified files.

The approach taken in this thesis is based on the list of DataComponents (see Section \ref{section:IntroDataComponent}). Each execution file 
carries a list of its own DataComponents and their properties to ensure
that the component are properly initialized the next time the execution is loaded. That list
can be loaded even if there are DataComponents contained in it that are not present in the current
runtime configuration. The Execution Manager will show a warning indicating that it doesn't know the given
component but proceed to load the rest of the execution. 

These DataComponents basically carry a list of the KiemProperties described in Section \ref{section:IntroKiemProperty}.
The properties can carry a list of (key, value) pairs which means they are suited well for storing simple information like the value
of a text field.

To solve the problem a new type of DataComponent will have to be constructed and registered through
the extension point in the Execution Manager. This ensures that the component can be added to any execution
file. The Execution Manager  ensures that all properties contained in our component will be saved with
the execution file and loaded the next time the file is opened. After that the configuration plug-in has 
find the component inside the DataComponent list, get the properties saved inside it
and set them inside the Execution Manager.

\subsection{Default Configuration}
\label{section:ConfConceptsDefaultConf}
\index{Default Configuration}
In order to provide a place to manage the default configurations the Eclipse preference page mechanism
will be used (see Section \ref{section:TechPreferencePage}).

The root page for the Execution Manager should contain the basic settings for the \ac{KIEM} itself like
the aimed step duration, timeout and the view elements of the Configuration plug-in.

\begin{figure}
  \centering
  \includegraphics[scale=.5]{MSPLayoutPreferencePage.jpg}
  \caption[Layout Preference Page by Miro Sp\"onemann]%
  {Layout Preference Page by Miro Sp\"onemann\protect}
  \label{fig:MSPLayoutPreferencePage}
\end{figure}

The next page that needs to be constructed will be used for managing the different schedules
and their editors. For that the LayoutPreferencePage by Miro Sp\"onemann
as seen in Figure \ref{fig:MSPLayoutPreferencePage} will be slightly adapted. The original preference page
displays a table where different layouters can be assigned priorities for different diagram
types. This is similar to the problem that needs to be solved in this thesis. The diagram types
can be directly mapped to the list of editors and the layouters will be replaced by the list of saved schedules.
That way the modified preference page can be used to assign priorities to schedules for
different editors. The priorities can then be used to sort the schedules matching
the currently opened editor.

The last page will be used to allow the user to set up his own properties and give them
default values.

The values entered in those pages will be stored inside the Eclipse preference
store (see Section \ref{section:TechPreferenceStore}).

\section{Easier Schedule Loading}
\label{section:ConfConceptsEasyLoading}
\index{Execution file}
To allow the user to easily load previously saved execution files two ComboBoxes will be introduces
to the Execution Managers tool bar.

One of them will display the list of recently used schedules the other one the
list of schedules that work for the currently active editor.

As soon as the user opens a new execution file through the normal workspace
view the \ac{KIEMConfig} plug-in will be notified of that event by the Execution Manager.
A new object will then be created which represents the execution file and contains its path.
This schedule object will also be added to the list of recently used schedules that is maintained through the 
use of the Eclipse preference store.

When the user selects one of the previously saved schedules in one of the
ComboBoxes the saved path will be retrieved and relayed to the Execution Manager to
load it in the hopes that the execution file is still in the same location and
wasn't deleted, renamed or moved by the user.

\chapter{Code Changes in the Execution Manager}
\label{chapter:KiemChanges}
Although the project attempts to realize most of the objectives without
changing the Execution Manager itself minimal adaptations were necessary.
This mostly involves adding new methods to the \ac{API} in order to
gain access to previously hidden properties.

Also some changes had to be performed where properties were loaded from hard-coded 
default values. These were refined and will now only be used if the \ac{KIEMConfig} 
plug-in is not registered to supply previously saved properties.

However all changes that were made to the \ac{KIEM} plug-in were merely additions
and won't break any plug-ins relying on the old implementation.

\section{Schema Files and Interfaces}
\index{Extension point}
In order to provide additional functionality for other plug-ins we choose the extension
point mechanism described in Section \ref{section:TechExtension}. This is done
is order to retain the old functionality of the plug-in while on the other hand giving
options to ask extending plug-ins for their contributions. 

The extension points are described in more details in the next sections. They consist
of a schema file for defining the extension point and an interface that contains the 
methods that extending components have to supply.

\subsection{Toolbar Contribution Provider}
\label{section:ToolbarContributionProvider}
\index{Toolbar Contribution Provider}
The purpose of the tool bar contribution provider is to allow other plug-ins to put
items onto the tool bar of the Execution Manager. 
There are two reasons for using the extension point mechanism
rather than making the tool bar available and have other plug-ins put their
contributions directly on it:
\begin{enumerate}
 \item At the moment the tool bar and all contributions are created dynamically. Switching
the entire native tool bar of the Execution Manager to adding the actions to the tool bar
through a predefined Eclipse extension point would require major code changes and
have major drawbacks.
 \item A programmatic approach gives control over the contributions to the Execution Manager.
This means that the order of the native Execution Manager buttons is always the same and in the
same place. It also means that the Execution Manager can choose to ignore contributions if the
tool bar gets too crowded.
\end{enumerate}
\begin{figure}[IToolbarContributor]
  \centering
  \includegraphics[scale=.3]{IKiemToolbarContributor.jpg}
  \caption[The interface for Tool bar Contribution Providers]%
  {The interface for Tool bar Contribution Providers\protect}
  \label{fig:IToolbarContributor}
\end{figure}

The schema file for components that want to add contributions to the tool bar is quite simple.
It only requires them to implement the interface shown in figure \ref{fig:IToolbarContributor}.
The implementing class provides an array with all ControlContributions they want to add to the tool bar.
A ControlContribution for a tool bar can be almost any widget like Labels, Buttons, ComboBoxes, ....

When the Execution Manager starts to build the views tool bar it will perform the following steps:
\begin{enumerate}
 \item The contributors will be asked for the list of controls that they want to contribute.
 \item That list will be added to the Execution Manager's tool bar.
 \item After that the Execution Manager will add its own controls to the tool bar.
\end{enumerate}
This order causes the tool bar to have the native elements always in the same order.
The contributed elements will be added from left to right in the order that they occur in the array. However there
is no guarantee on the order in which the extending plug-ins are asked.
Figure \ref{fig:ToolbarWithContributions} shows the Execution Manager tool bar with the two combo boxes belonging to \ac{KIEMConfig}
contributed through the extension point. Figure \ref{fig:ToolbarWithoutContributions} shows the same tool bar without
the contributions.
\begin{figure}[ToolbarWithContributions]
  \centering
  \includegraphics[scale=.4]{ToolbarWithContributions.jpg}
  \caption[The Execution Managers Tool bar with two contributed ComboBoxes]%
  {The Execution Managers Tool bar with two contributed ComboBoxes\protect}
  \label{fig:ToolbarWithContributions}
\end{figure}
\begin{figure}[ToolbarWithoutContributions]
  \centering
  \includegraphics[scale=.4]{ToolbarWithoutContributions.jpg}
  \caption[The Execution Managers Tool bar without contributions]%
  {The Execution Managers Tool bar without contributions\protect}
  \label{fig:ToolbarWithoutContributions}
\end{figure}

\subsection{Configuration provider}
\label{section:ConfigurationProvider}
\index{Configuration Provider}
The purpose of the configuration provider is to allow internal attributes of the
Execution Manager to be stored in another plug in. 

This is achieved by another extension point to allow any plug-in to listen to changes
in the Execution Manager's attributes. It also means that there may be multiple
plug-ins that provide values for properties and not all plug-ins may have the value for
a property needed by the Execution Manager. Through the plug-in mechanism the \ac{KIEM}
can ask all providers for values and choose the one he would like to use.

\begin{figure}[Configuration Provider Interface]
  \centering
  \includegraphics[scale=0.3]{IKiemConfigurationProvider.png}
  \caption[The Interface of the Configuration Provider]%
  {The Interface of the Configuration Provider\protect}
  \label{fig:UMLConfigurationProvider}
\end{figure}

The two methods from the interface shown in Figure \ref{fig:UMLConfigurationProvider} work
in the following way:

\subsection{Event Listener}
\label{section:EventListener}
\index{Event Listener}
The main purpose of the Event Manager is to inform DataComponents of events
happening in the Execution Manager during execution. This behavior has been modified to include 
events that occur while the execution is not running. This modification lead to the 
creation of another extension point in order to allow other plug-ins to be notified
on any of these events as well. The classes implementing the interface 
(see Figure \ref{fig:UMLEventListener}) required by this extension point will be notified
on any event that happens inside the Execution Manager.

\begin{figure}[Event Listener Interface]
  \centering
  \includegraphics[scale=0.3]{IKiemEventListener.png}
  \caption[The Interface of the Event Listener]%
  {The Interface of the Event Listener\protect}
  \label{fig:UMLEventListener}
\end{figure}

\begin{description}
 \item \textbf{int provideEventOfInterest()} : This method is directly derived from the method with 
the same name in the AbstractDataComponent class of the \ac{KIEM} plug-in. It is called by the
EventManager to determine which events the implementing class is interested in.
This is done to improve efficiency and not flooding components with events they are not interested in.

Based on the response, the EventManager puts the component into lists along with the DataComponentWrappers
already inside.
 \item \textbf{void notifyEvent(KiemEvent event)} : This method is called by the EventManager when 
something happens inside the Execution Manager that the implementing classes might be interested in.
\end{description}

\section{KIEMPlugin.java}
\label{section:ConfChangesKiemPlugin}
The main Activator class contains almost the entire \ac{API} of the \ac{KIEM}.
Therefore any additions to that has to performed in this class which means that
most of the adjustments were made here.

\subsection{Listener}
The following methods were added to communicate with the plug-ins registered through
the ConfigurationProvider extension point (see Section \ref{section:ConfigurationProvider}).
\begin{description}
 \item \textbf{notifyConfigurationProviders(String propertyId, String value)} : This method can be called by any class
inside the Execution Manager itself. It should be called when the user changes a property through any of
the elements on the \ac{GUI}. The method will then inform all listeners that the property identified by the
given identifier was changed to the new value.
 \item \textbf{String getPropertyValueFromProviders(String propertyId)} : This method allows the Execution Manager to
retrieve a previously saved value. The \ac{KIEM} will ask all plug-ins registered on the extension point if they
can provide a value for the given identifier. Plug-ins that can't provide the value will indicate this by throwing
an Exception. The \ac{KIEM} will then take the first value he receives without getting an Exception and assign it
to the internal property.
 \item \textbf{Integer getIntegerValueFromProviders(final String propertyId)} : This method is a convenience method for
the one described above. It will try to parse an integer from the retrieved String and return it or return null
if no integer could be parsed.
\end{description}

\subsection{Getters and Setters}
An example for the use of the methods described in the last section can be found in the Getters and Setters for
the different properties in the Execution Manager (for an example see Figure \ref{list:getterAndSetter}). 
These were changed in order to use the new methods but are
still able to fall back on hard-coded default values if no configuration plug-in is registered.

\listingjava
\showlistingex{code/getterSetter.txt}
{Java}
{Example of modified Getter and Setter}
{list:getterAndSetter}
{t}


\subsection{Open File}
The method that takes care of loading an execution file was split. This was done to allow
other plug-ins to pass an IPath object directly to the method and perform a load of that file without
having to go through the \ac{UI}. This method was also shifted around a little in order to detect
missing execution files before the load enters the UIThread. This was necessary to make is possible for
the callers of the method to catch the resulting exception.
The method also had to be modified in order to be able to take files that are not inside the workspace
but were added through an extension point. The changed part of the openFile method is shown in 
Listing \ref{list:openFileMethod}.
The last change to that method concerns the event listener. When the user opens a file through the
Eclipse workspace without using the \ac{KIEMConfig} plug-in the plug-in still has to be informed.
This happens through the use of the Event Manager that notifies all listeners upon the successful
completion of the loading operation.

\listingjava
\showlistingex{code/openFile.txt}
{Java}
{The head of the modified openFile() method}
{list:openFileMethod}
{t}


\section{KIEMView}
\label{section:ConfChangesKiemView}
The changes described in Section \ref{section:ConfChangesKiemPlugin} were mostly concerned with the
configuration management and loading of new execution files. This section will mostly deal with the changes
that were necessary to enable the adding of new items to the tool bar.

\begin{itemize}
 \item changes to tool bar creation
 \item provide means to set view as dirty when changes happen
 \item refresh method to reload values, not needed before because no change except by user
\end{itemize}
\chapter{Kiem Configuration Plug-in}
\label{chapter:KiemConfig}
This chapter describes the contents and functionality of the newly created
plug-in to solve the problems described in Chapter \ref{chapter:ConfTask}.
A new plug-in was created in order to improve modularity within the \ac{KIELER} framework.
Putting the code into the \ac{KIEM} plug-in itself would have meant that
there would have been no way to separate the two projects.

The sections in this chapter describe the different parts of the \ac{KIEMConfig} plug-in.
The whole plug-in is structured according to the Model-View-Controller pattern.
The first section will describe the data storing classes which constitute the model.
The second section will describe the different manager classes which are essentially the
controller of the entire plug-in. This section will also look at the \ac{API} that the
\ac{KIEMConfig} plug-in provides to other plug-ins.
The last section will describe the classes that render the preference pages and other
view elements.

\section{Data Classes and Utilities - the Model}
\label{section:ConfModel}
This section will describe the different classes that are responsible for storing all
data that the plug-in needs at runtime.


\subsection{ConfigDataComponent}
\label{section:ConfigDataComponent}
\index{ConfigDataComponent}
This extension to the AbstractDataComponent of the \ac{KIEM} is responsible for solving
the problem described in Section \ref{section:ConfTaskConfig} and implements the behavior
described in Section \ref{section:ConfConceptsConf}. The component is a DataComponent
like all others used in the \ac{KIEM}. It is registered through the extension point
that allows new DataComponents to appear in the list of available components.
However unlike the usual DataComponent that is responsible for simulating a model during
an execution run its main function is to store the configuration of the \ac{KIEM}.

Like all other DataComponents the ConfigDataComponent contains an array of
KIEMProperties. These properties contain a String key which should be non-null and unique and 
a value which can be of various types. However for the purpose of storing configuration
elements only the String value will be used.

The new DataComponent also provides additional methods in order to make accessing and manipulatin the
array more convenient:
\begin{description}
 \item \textbf{KiemProperty findProperty(String key)} : This method iterates through the array and attempts
to find the KiemProperty that has exactly the provided key. Since the keys are assumed to be unique the first
match is returned by this method. If there is no property with the given key the method will throw
an Exception.
 \item \textbf{void removeProperty(String key)} : This method attempts to remove the property identified by
the given key from the array. It does this by converting the array to a list, locating and removing the
specified property and then converting the list back to an array. This procedure may not be as efficient
as manually constructing the new array but it still performes the operation in linear time. Furthermore
it makes the method easier to understand than the alternative.
 \item \textbf{KiemProperty updateProperty(String key, String value)} : This method updates the property 
identified by the key with a new value. It first checks if the property already exists and if it does its value
is updated. If a property with the specified key doesn't exist a new one is created and the provided value
stored inside.
\end{description}

In addition to those methods the ConfigDataComponent also keeps a reference to its DataComponentWrapper (see Section
\ref{section:IntroDataComponentWrapper}). This is necessary in order to retrieve the properties from the wrapper
right after the execution file was loaded and to write them back into the wrapper before the file is saved.

TODO:
\begin{itemize}
 \item can be added and removed by the user to update old files or downgrade new ones
 \item used as current/default configuration
\end{itemize}

\subsection{EditorDefinition}
\label{section:EditorDefinition}
\index{EditorDefinition}
The EditorDefinition class is responsible for storing information about the editors that are known to
the \ac{KIEMConfig}. Each instance of this class stores the information about a single editor. This is
necessary in order to successfully operate a list of execution files that work for the currently active
editor.
\begin{description}
 \item \textbf{String editorId} : The identifier for the given editor. This attribute is a unique non-null String
by which any editor can be identified. For example the standard Java editor has the id \textit{org.eclipse.jdt.ui.CompilationUnitEditor}.
 \item \textbf{String name} : The name of the editor. This is the human readable name given to the editor
by the plug-in that defines the editor. Storing this attribute may seem redundant since the names of
the editors can be retrieved through an Eclipse mechanism if the editor id is known. However there is no
guarantee that a previously saved editor id exists in the currently active application in which case the name
of the editor can't be retrieved.
 \item \textbf{boolean isLocked} : This attribute is responsible for showing that the editor can not be removed.
The reason that an editor might become read only will be explained in Section \ref{section:DefaultSchedule}.
\end{description}

\subsection{ScheduleData}
\label{section:ScheduleData}
\index{ScheduleData}
The ScheduleData class is responsible for tracking the different execution files that are known
to the \ac{KIEMConfig}. A ScheduleData object is the representation of a single execution file. 
These objects are used to main he lists of recently used schedules and of those that match the 
currently opened editor. It contains the following attributes:
\begin{itemize}
 \item The most important attribute is the path at which the execution file that this instance
should represent is located. The path is used to trigger the loading of the file inside
the Execution Manager. It is also used to determine whether a newly loaded execution file
is already known. The path also doubles as the unique identifier for the schedule since there
can't be two files at the same physical location.
\item The ScheduleData object also stores a list of priorities for all known editors. This is
necessary in order to determine whether or not a given schedule can be used with the currently
opened editor and which position it should have in an ordered list. To make accessing and manipulating
this list easier it simply uses an instance of the ConfigDataComponent. The component already has
methods for accessing the array inside and can be easily stored and loaded.
 \item Like the EditorDescription a ScheduleData also contains a boolean \textbf{isLocked}. ScheduleData
object with that attribute set to \textit{true} can't be modified or removed (see Section \ref{section:DefaultSchedule}).
\end{itemize}

MAYBE TODO:
\begin{itemize}
 \item getting/setting supported priority, adding/removing editors
\end{itemize}

\subsection{Tools}
\label{section:Tools}
The Tools class holds a host of useful methods and attributes that are used in various parts of the plug-in.

\subsubsection{Attributes}
\label{section:ToolsAttributes}
First of all it contains messages and tooltips that are used in more than one class.
This ensures that the appearance of the different messages is unified across the entire plug-in. It also
makes it easy to change these messages or combine different partial messages to new ones.

The class also holds the different identifiers for the properties that are used in the plug-in. This is done
to avoid bugs due to mistyping an identifier which is likely to happen if it is stored in two different places.

\subsubsection{Methods for Parsing and Serialization}
\label{section:ToolsMethodsParsing}
All of the manager classes in the \ac{KIEMConfig} need to save their properties into the Eclipse preference store.
In order to have the information stored in a structured way an XML like format was chosen. As this requires the keys and values
to be formatted in a certain way the Tools class provides methods to format the Strings in the required way.
\begin{description}
 \item \textbf{String putValue(String key, String value} : Converts the (key, value) pair into a formatted String for saving
into the Eclipse preference store. The resulting String has the following format: 
\textit{<[key]>[value]</[key]>}.
 \item \textbf{String putProperty(KiemProperty property} : Convenience method for transforming a KiemProperty object into a
formatted String. This method exists because most of the items serialized in this way are of that type. The resulting String
has the following format: \textit{<KIEM_PROPERTY> <Key> [property.key] </Key> <Value> [property.value] </Value> </KIEM_PROPERTY>}.
\end{description}

The methods described above provide all the necessary facilities for the \ac{KIEMConfig} to save its preferences
into the Eclipse preference store. In order to retrieve these properties the Tools class provides another set of
methods. These methods take an input String and try to parse the saved properties.
\begin{description}
 \item \textbf{String getValue(String key, String input)} : This method retrieves the value enclosed by tags with
the given key. The retrieved value can either be an atomic String that can directly be assigned to a property or
another series of values enclosed in their tags. The method will always look for the outermost tags inside the
input String. The method returns null if there are no tags with the provided key inside the input String.
 \item \textbf{KiemProperty getKiemProperty(String input)} : This convenience method tries to retrieve the 
(key, value) pair that constitutes a KiemProperty object from an input String.
 \item \textbf{String[] getValueList(String key, String input)} : Since there sometimes is the need to store an entire
list of entities the Tools class provides a method to convert an entire list back to the individual Strings.
The method iterates over the input String and extracts all elements that are enclosed in tags with the specified key.
\end{description}


\subsubsection{Methods for Dialogs}
\label{section:ToolsMethodsDialogs}
The Tools class also contains methods for easily displaying error and warning dialogs.
These methods take the information, add the own plug-in id and forward the information to the 
error handling facilities inside the Execution Manager itself.


\subsection{MostRecentCollection}
\label{section:MostRecentCollection}
The MostRecentCollection is a new collection type that is can be used for simulating the 
behavior found in 'Open recent' menu item of almost any text editing application.
To avoid the list growing too long it can be given a maximum capacity. After that capacity
is reached the oldest entry will be deleted when a new one enters the list.
The default implementation of the collection uses an ArrayList to store the data but any
other list works as well. Most operations are directly delegating to the operations of the 
underlying List. 
The only exception is the add(item : T) method that works in a different way:
\begin{enumerate}
 \item It checks if the item is already in the list and removes it. This is done to ensure
that already added items don't appear twice in the list.
 \item It adds the item at the highest index to the end of the list and increments the index of all other items.
 \item The element at the head of the list is overridden by the new item.
 \item Optionally the last item is removed if the list has grown beyond the capacity.
\end{enumerate}
The collection also provides an additional method that is used to replace an item in
the list by another one. This is used when files are renamed and the name of the ScheduleData inside
the list has to be updated.

This collection is used to track the most recently used schedules and display them
in the corresponding ComboBox.



\section{Manager Class - the Controller}
\label{section:ConfController}
The manager classes are responsible for the control flow inside the plug-in. They gather information
from the view, the Eclipse preference store and the Execution Manager and create and update a
model using the classes described in Section \ref{section:ConfModel}. There are multiple managers
each with a different task:
\begin{itemize}
 \item The \textbf{Configuration Manager} is responsible for maintaining the configuration saved in
each execution file and the default configuration saved in the preferences store.
 \item The \textbf{Schedule Manager} is responsible for keeping track of the different
execution files and updating the information inside the ScheduleData objects.
 \item The task of the \textbf{Editor Manager} is to
\end{itemize}



\subsection{Abstract Manager}
\label{section:AbstractManager}
All of the managers share some common features that each of them must provide. Some of those
features are handled almost the same or exactly the same in each manager. This lead to the creation
of an abstract super class for all managers that takes care of the basic tasks.

The first task is to allow other classes to register as a listener to the manager. Some of the classes
in the \ac{KIEMConfig} have to perform updates when a value inside the model changes. It is the managers
responsibility to inform the listeners when such a change was completed successfully.

The second task is to provide the subclasses with facilities to easily access the Eclipse Preference Store.
Whenever a value is requested by any part of the controller or another plug-in and a manager didn't access
the preference store yet it has to gain access to the store and retrieve the information belonging to it.
Furthermore when the user explicitly wants to save the preferences or the workbench is shutting down the
data contained in the model has to be saved into the Eclipse Preference Store. The Abstract Manager provides
some methods seen in Figure \ref{fig:AbstractManagerUML}. (For an example of a saved configuration see Appendix \ref{section:AppendixSavedConf}).


TODO:
\begin{itemize}
 \item load/save through the plug ins preference store
 \item add/remove/notify listeners
\end{itemize}

\subsection{Configuration Manager}
\label{section:ConfigurationManager}
\index{Configuration Manager}
\begin{itemize}
 \item manages current/default configuration, supply filtered lists for preference components
 \item find current configuration and wrapper in data component wrapper list 
 \item get values for properties (where to look, ignored keys, save to current when found)
  \begin{itemize}
   \item check if there is a current configuration, check if the key is not on the ignored key list, try to find in current configuration
   \item on failure: try to find in default configuration, update current configuration, return found value
   \item on failure: if default value supplied, update default configuration, update current configuration, return default value
   \item on failure: throw exception
  \end{itemize}
 \item add/remove properties, update values
 \begin{itemize}
  \item try to update saved values
  \item if it doesn't exist create it
 \end{itemize}
 \item restore default values
\end{itemize}

\subsection{Schedule Manager}
\label{section:ScheduleManager}
\index{Schedule Manager}
\begin{itemize}
 \item all schedule data
 \item track recently used schedule Ids
 \item ask KIEMPlugin to load a saved schedule, deal with error
 \item add/remove schedules, update schedule priorities
 \item handle user load/save events
\end{itemize}


\subsection{Editor Manager}
\label{section:EditorManager}
\index{Editor Manager}
\begin{itemize}
 \item all editor definitions
 \item add/remove/find editors
 \item default editors 
\end{itemize}

\subsection{Contribution Manager}
\label{section:ContributionManager}
\index{Contribution Manager}
contribution manager: create combos, save visibility

\subsection{Property Usage Manager}
\label{section:Property Usage Manager}
\index{Property Usage Manager}
property usage manager: track all keys where default config should be used rather than current


\subsection{Implementing Classes}
\begin{itemize}
 \item tool bar provider: link to contribution manager, forward array of contributions
 \item configuration provider: link to configuration manager, forward requests
 \item event listener: handles events received by the configData component 
    listen to load/save events, can be disabled when KIEMConf is about to trigger load/save
\end{itemize}

\section{Preference Pages - the View}
\begin{itemize}
 \item place to easily configure settings, all KIELER preferences in one place
 \item integrated into eclipse look and feel
\end{itemize}

\subsection{Configuration page}
\begin{itemize}
 \item changing default configuration for internal properties
 \item check boxes for changing visibility of the combos
\end{itemize}

\subsubsection{User defined properties page}
\begin{itemize}
 \item adding/removing properties
 \item modified from msp, table view with providers
\end{itemize}

\subsubsection{Property usage dialog}
This dialog shown in figure \ref{fig:PropertyUsageDialog} is used for selecting which properties should always be taken
from the default configuration rather than the configuration component contained
in every .execution file.
The dialog used for this is a ListSelectionDialog which just receives the list of
all keys as input and the list of PropertyKeys from the PropertyUsageManager as default selection.
After the user is finished with selecting attributes and hit the 'Ok' Button the dialog
passes the new list of selected items back to the PropertyUsageManager.
\begin{figure}[PropertyUsageDialog]
  \centering
  \includegraphics[scale=.5]{PropertyUsageDialog.jpg}
  \caption[Property Usage Dialog]%
  {The Property Usage Dialog\protect}
  \label{fig:PropertyUsageDialog}
\end{figure}

\subsection{Scheduling page}
This preference page is used to manage the schedules and the editors that they belong to.
This page is basically a modified version of the LayoutPrioritiesPage by msp.

\begin{itemize}
 \item table with schedules / editors and their priorities
 \item modified from msp , editors = diagram types, schedules = layouters
 \item modify priorities
 \item add/remove editors, remove schedules, selecting default editor
\end{itemize}

\subsubsection{Adding and removing editors, Selecting a default editor}
On the scheduling preference page there are routines for adding and removing
editors as well as selecting a default editor.
All of these actions use the same basic method for displaying an ElementListSelectionDialog \ref{fig:EditorSelectionDialog}
that takes a list of editor ids and returns the one selected by the user.
\begin{itemize}
 \item The editor adding dialog gets a list of all editors currently registered on the
 active workbench. The user can select a single editor which is then added to the table.
 \item The editor removal dialog gets a list of all editors currently available for 
 assignment of support properties. The editor selected by the user is removed from the table.
 It is also removed from all schedules. This is done to prevent the schedule objects from growing
 to monstrous size over time when editors are getting added and removed.
 \item The default editor selection dialog gets the same list as the removal dialog. The selected
 editor is then set as default editor. The default editor is used when there is no currently active editor on the workbench.
  It is used:
  \begin{enumerate}
   \item to determine which editors to show in the Matching combo box.
   \item when a new schedule is created as an editor id.
  \end{enumerate}
\end{itemize}
\begin{figure}[EditorSelectionDialog]
  \centering
  \includegraphics[width=1\textwidth]{EditorSelectionDialog.jpg}
  \caption[Editor Selection Dialog]%
  {The Editor Selection Dialog\protect}
  \label{fig:EditorSelectionDialog}
\end{figure}

\subsection{Default Schedule}
\label{section:DefaultSchedule}
\index{Default Schedule}
\index{Extension point}
\chapter{Conclusion}
\label{chapter:ConfConclusion}
As stated in Chapter \ref{chapter:ConfTask} the problem consists of two parts:
\begin{enumerate}
 \item Find a way to add configurations to the existing execution files. Additionally find
a way to allow the user the set up default preferences.
 \item Make it easier to load previously saved schedules without having to locate
the execution file in the workspace.
\end{enumerate}
The following sections will summarize the results and provide some ideas for
future work.

\section{Results}
\begin{itemize}
 \item problems solved
\end{itemize}

\section{Further Improvements}
Although all initial goals of this thesis were more than met there are still
some features that could not be added due to the lack of time.
\subsection*{Eclipse Runtime Mechanism}
The Eclipse framework provides a very comprehensive system to run different
modules. This is used to execute Java programs or to start a new Eclipse application
but there are a variety of other applications as well.

The entire Execution Manager could be refactored into using that runtime mechanism instead
of setting up a run through the now present \ac{KIEM} view. This means that the table
that shows the DataComponents has to be moved to a new runtime page. 

The controls for pausing, resuming, stopping and stepping through the execution have to
be moved somewhere else as well, possibly the DataTable.

\subsection*{Improve Storage Options}
Currently DataComponents as well as the preference mechanism only allow the use of
Strings to store the preferences. This means that all primitive data types can more
or less be stored by conversion to a String. However more complex objects can't be stored
without serializing them into a String and parsing them again on load.

A future project could try to find a way to overcome that limitation by allowing
objects that implement the Serializable interface to be stored as well.

\part{Automated Execution}
\chapter{Used Technologies}
\label{chapter:AutoTechnologies}
In addition to the technologies used in the first part of this
thesis (see Chapter \ref{chapter:ConfTechnology}) other Eclipse
technologies will be used as well.

The next sections will describe the technologies and give some
examples of their usage in the standard Eclipse application.
The technologies described here are the following:
\begin{description}
 \item The Job: The Eclipse Job is a mechanism for very long running tasks.
 \item The Wizard: The wizard is a method for helping the user to set up complex tasks.
\end{description}

\section{The Job}
\label{section:AutoTechJob}
The Eclipse Job \ac{API} provides the means to schedule very long running tasks.
It uses a Thread to run the actual task and contains a ProgressMonitor to show
the progress of the task. Since it is a task that can run independently of the
current state of the workspace it can also be run in the back ground if the user
desires it.
An example for the use of jobs in the normal Eclipse architecture
is the \ac{SVN} commit operation seen in Figure \ref{fig:SVNCommit}. An \ac{SVN} commit involves
sending a possibly large amount of files to a remote location over a network connection. That
operation might take a very long time. Furthermore there is no reason to prevent the user from
continueing to work while the files are send which means that the job can be run in the back ground.
\begin{figure}
  \centering
  \includegraphics[scale=.4]{SVNCommit.png}
  \caption[The SVN commit job]%
  {The SVN commit job\protect}
  \label{fig:SVNCommit}
\end{figure}

\section{Eclipse Wizards}
\label{section:AutoTechWizards}
\index{Wizard}
Wizards are used to guide the user through the process of creating complex items by taking
the information in a structured way and then generating the item from it. A wizard is 
basically a multi-page dialog with each page representing one step in the creation of the 
desired item.

One example inside the Eclipse Architecture is the Java Class Creation Wizard (see Figure \ref{fig:ClassWizard})
In theory it is possible to open a text file and enter all the information manually.
However if the wizard is used the user only has to select the class he wants to extend
and the interfaces he wants to implement, activate one check box and then the wizard will
create the class body, all required methods and comments for each element (see Listing \ref{list:classCreationGenerated})
This makes it very easy for even inexperienced users to create new classes without knowing
the exact syntax.
\begin{figure}
  \centering
  \includegraphics[scale=.4]{ClassWizard.png}
  \caption[The Class Creation Wizard]%
  {The Class Creation Wizard\protect}
  \label{fig:ClassWizard}
\end{figure}

\listingjava
\showlistingex{code/newClassGenerated.txt}
{Java}
{Code generated by the wizard}
{list:classCreationGenerated}
{t}


\chapter{Problem Statement}
\label{chapter:AutoTask}
The objective of this project is to find a way to automate the execution
runs of the Execution Manager as described by Christian Motika in his
diploma-thesis\cite{cmot-dt}.

Currently the Execution manager works in a way that the user manually sets up
a new execution run or loads a saved execution file. The DataComponents then have 
to gather all information they need themselves like model files, trace files and so on.
Since there is no generic way to do that, this information is either hard-coded into
the components or entered manually through the properties.
After that the user has to manually control the execution. The execution runs until
the user or a component stops it. The user then has to manually set up another
execution run, possibly even rewriting his components if the model files
and trace files are hard-coded or manually change the properties.
This is very unsatisfactory if you have a large number of model files that
should be tested with a one or more execution files and possibly hundreds of trace files
(see Figure [TODO]).
% ask ctr for screen shot of directory perhaps
Performing runs like that manually is completely out of the question as even
with the automation in place it would take several hours.

The task resulting from this problem can be broken down in 4 parts which are explained in detail in
the following sections:
\begin{enumerate}
 \item The setup of an automated run by the user.
 \item The input of all the necessary information.
 \item The control flow of the automated run.
 \item The gathering of information after the run has finished and the display 
of that information.
\end{enumerate}

\section{Setting up a Run}
\label{section:AutoTaskSetup}
The first objective is to find an easy way for the user to efficiently set up an
automated run. This involves selecting the model files and execution files
needed for the automation as well as entering initial properties.

\section{Input for the Automation}
\label{section:AutoTaskInput}
The second objective is to enable the components to receive inputs.
Each component should receive all information it needs prior to each execution
run in order to make the components more dynamic. This mechanism would ensure
that components can be written in a more generic way than is currently possible.
We will have to define an \ac{API} for this information passing process as well as
an \ac{API} to trigger an automated execution from other plug-ins.


\section{Automate the execution}
\label{section:AutoTaskExecution}
The third objective is to automate the control flow of the execution itself. This would
involve the following:
\begin{enumerate}
  \item Loading the desired execution files, model files and trace files.
  \item Determining how many steps should be performed and running the execution up the desired step.
  \item Gathering the information produced by the components.
  \item Properly shut down the execution so that a new one can be started.
\end{enumerate}


\section{Output execution results}
\label{section:AutoTaskOutput}
The last objective is to display the information in a meaningful way.
This should involve at least two methods of output:
\begin{enumerate}
  \item A formatted string possibly in an XML fashion that can be parsed and
  used by other plug-ins for automated analysis.
  \item Some graphic component that will display the information in a way that is
  easy to read for most users.
\end{enumerate}


\chapter{Concepts}
\label{section:AutoConcepts}
This chapter presents the conceptual solution to the problems described in Chapter \ref{chapter:AutoTask}.

It will follow the same structure to make it easier to follow. This means that the chapter will start
by offering a few possible options for the user to set up an automated execution. It will continue
by providing alternatives of how DataComponents receive the necessary information prior to each run.
The next section explains how the actual control flow throughout the automation will be handled.
In the last section some possibilities for user-friendly output are presented.

\section{Setting up a Run}
\label{section:AutoConceptsSetup}
There are several possibilities of how to solve the problem of accumulating
large amounts of information prior to a long running action.

The first possibility would be to have the user enter the paths to the 
necessary files in text files, parse those files and start a run with
the parsed information. While this is a good method for performing
static runs from a console environment it has several disadvantages
inside the \ac{GUI} of an Eclipse \ac{RCA}:
\begin{itemize}
 \item Manually entered file names in a text file are prone to have erroneous information.
It is very hard to manually enter the correct file name of any file and the entered location
only works on one \ac{OS}. Aside from that it takes a long time to manually enter the possible vast
amount of files used.
 \item There is no way to quickly adjust the file if other models or execution files should be used.
 \item It also means more files cluttering up the workspace.
 \item It is not very intuitive and the user has to know the exact syntax that the execution needs.
\end{itemize}

Another approach is the selection of the files through a dialog.
Here the first option is to write a new dialog from scratch. While this option
ensures flexibility since only the elements that are really needed are on it in
exactly the way they are needed there are still some disadvantages:
\begin{itemize}
 \item It involves a lot of work since every widget has to be manually placed on the dialog.
 \item It involves even more work to get the layout of the dialog just right.
\end{itemize}

A more comprehensive approach would be to use one of the dialogs provided by Eclipse specifically the wizard type dialog.
Eclipse itself uses a host of wizards as explained in Section \ref{section:AutoTechWizards}.
The wizard has several advantages over the other methods explained above:
\index{Wizard}
\begin{itemize}
 \item Even inexperienced users can be guided to set up a valid execution run.
 \item The entered information is most likely valid since the wizard only displays valid files.
 \item It is quicker to program and easier to adjust than any of the other methods.
\end{itemize}


\section{Input for the Automation}
\label{section:AutoConceptsInput}
\index{DataComponent}
In order to send information to the DataComponents\footnote{Section \ref{section:IntroDataComponent}}
the first decision must consider the form of the information that will be supplied.
The chosen form is that of a list of (key, value) pairs. This format supplies the greatest
flexibility while still being very generic and simple to read and write on.
This list of properties will at least include the path to the model file\footnote{Section \ref{section:IntroModelFile}} in order
for components to be executed with several different model files without having
to alter the code between runs.

The next decision involves how the components are getting the information.The first possibility 
would be to have the component ask the plug-in for the information
in question. The upside of this would be that components are sure to get all the information
they need before the execution can start since they can keep asking for it. However this would
likely mean that the component has to poll multiple times as it has no knowledge about when
the required information will be available which constitutes additional workload. Furthermore
this situation would likely mean that multiple components might request information
at the same time. This means that there would be the need for substantial synchronization mechanisms
to ensure consistency of data.

Therefore the way chosen in this thesis is that the Execution Manager will inform interested components
about all properties that were accumulated and then starts the execution run. This ensures
that a run is started in any event and keeps communication between the components and the manager simple.

\section{Automate the Execution}
\label{section:AutoConceptsExecution}
As a basic control flow for the automation the following procedure is chosen:
\begin{enumerate}
 \item The automation will iterate over all supplied execution files. These are likely the most time consuming
to load which means loading an execution file multiple times should be avoided.
 \item With each execution file the automation will iterate over all model files. Model files are costly to load
as well however each load of an execution file would mean that all DataComponents would not be able to store their
saved properties either way which in turn means they would have to reload the model anyway. For this reason the model
files will be loaded once for every execution file.
 \item With each model file the automation will perform multiple runs. That means starting a run in the execution manager
and performing a certain number of steps and the stopping the execution again. With each run the components get the chance
to execute a few steps with different properties, for example trace files. These runs will be called iterations from this
point forward.
\end{enumerate}

Automating the execution itself requires the plug-in to interact with the Execution Manager.
There is already an \ac{API} defined for loading an execution file by supplying a path so that
is what will be used in this project.
Then it is necessary to initialize the execution and step through it using the \ac{API}
methods provided in the Execution Manager. For this the EventListener extension point\footnote{Section \ref{section:EventListener}} of the 
Execution Manager can also be used in order to determine when a step has finished executing and
a new one can be dispatched.
After the execution is finished all components should be called again to be given
a chance to provide information for the display in the next step. This information
will be gathered in the same form and way as described in Section \ref{section:AutoConceptsInput}.

\section{Output Execution Results}
\label{section:AutoConceptsOutput}
On the subject of displaying the information several options are available.

The first option would be to open a dialog once the execution has finished
and display the results in a tabular manner. This has the advantage that the
users attention is immediately caught when the run finishes. However pop-up boxes
should be used only when something very important happens and only with small messages
as they tend to interrupt the users work flow. Aside from that the user might want
to look at the results from the execution and compare them with the actual model.
An opened dialog is usually something sitting in front of the rest of the \ac{IDE}
and that user wants to get rid of as fast as possible.

The option taken in this project will utilize the view mechanism provided by Eclipse. As described
in Section \ref{section:ConfTechEclipse} a view is used to display content that was created
elsewhere. Therefore it is the logical choice for displaying the information created by the automated
execution. This approach ensures that the user can place the displayed results where he wants them to be.
It also makes it possible for the user to run an execution multiple times and compare the
results by having them displayed in multiple views or next to each other in the same view.
Another advantage of the view concept is that it provides a tool bar for adding actions.
The user might want to control the automated run during its execution or interact with
the displayed results. While a static dialog would have difficulties providing the control elements
for those actions a view can easily display them in the tool bar.

\chapter{Code Changes in the Execution Manager}
\label{chapter:AutoKiemChanges}
In order for the \ac{KIEMAuto} to operate successfully there were
no changes necessary beyond the ones described in Chapter \ref{chapter:KiemChanges}. However the 
newly created plug-in uses many of these features and could not achieve its objectives without them.

The \ac{KIEMAuto} makes use of the event listener extension point 
(see Section \ref{section:EventListener}) in order to get 
notified about events occurring during the current execution run. This involves receiving information
about the completion of a step or user- and error-triggered termination or the currently active execution.

It also uses the tool bar contributor extension point (see Section \ref{section:ToolbarContributionProvider})
to add new control elements to the Execution Managers tool bar.

\chapter{The Automated Executions Plug-in}
This chapter describes the Automated Execution plug-in that is responsible for handling
the setup, control flow and display of an automated execution run.
Although the plug-in is structured according to the model-view-controller pattern this
is not the structure that this chapter will to describe it. Instead this chapter will
follow the structure already used in the previous chapters. Which means the chapter will
consist of four parts:
\begin{enumerate}
 \item The setup of an automated execution run with a description of the wizard.
 \item The input to the automated execution. This section will describe the newly created interface.
 \item The control flow of the automation with the Automation Job and the Automation Wizard.
 \item The output of the automation run. In this section the new view and the results structure will
be described.
\end{enumerate}


\section{Automation Setup - The Wizard}
\label{section:AutoWizard}
\index{Wizard}
As described in Section \ref{section:AutoConceptsSetup} an Eclipse wizard will be used to set up the 
automated run. For easy access the button for opening the wizard (
\includegraphics[scale= 0.7]{kiemAutomated.png})
is located on the tool bars of both the Execution Manager and the Automated Execution view.

The Automation Wizard consists of two pages:
\begin{enumerate}
 \item The File Selection Page for selecting all files that should be involved in the automated run.
 \item The Property Setting Page for defining custom properties that the components should receive prior to
each iteration.
\end{enumerate}


\subsection{File Selection Page}
\label{section:FileSelectionPage}
\begin{figure}[File Selection Page]
  \centering
  \includegraphics[scale=.4]{FileSelectionPage.png}
  \caption[The Wizard Page for selecting the input files for an automated run.]%
  {The Wizard Page for selecting the input files for an automated run.\protect}
  \label{fig:FileSelectionPage}
\end{figure}
The File Selection Page shown in Figure \ref{fig:FileSelectionPage} is used to select the model 
files and execution files that should be used for the automated run.

Since Eclipse already provides a variety of pre-made wizard pages it can be avoided to write a page for a
complex task like this from scratch. The wizard that can be modified to fit the needs of the task at hand
is the standard Eclipse Resource Import Wizard. It is normally used to select a number of files and folders
for import into the workspace. It provides a structured view where entire folders can be selected, files
can be filtered by their type and additional space is available for other buttons. In this project the files
will be 'imported' into the automated execution. This is similar enough to make it possible to use the wizard
with a few modifications and extensions.

As it should be as easy as possible to set up a run for the user it would be desirable that he doesn't need
to select all the files each time the wizard is opened. For this reason the selection will be saved into
the Eclipse preference store every time the wizard is closed. The next time the wizard is opened the selection
only has to be retrieved from the preference store and passed to the Resource Import Wizard super class.

The \ac{KIEMConfig} allows for execution files to be linked into the workspace through an extension
point. This is a useful feature for adding factory defaults and as such \ac{KIEMAuto} naturally wants to
be able to use these execution files as well. However since they these files are not in the workspace they
can't be selected through the main area of the wizard page. In order to select these files anyway a simple
list selection dialog can be accessed through the button at the bottom of the page. The dialog displays all
schedules imported through the extension point and allows the user to select any number of them.

The hard part is how to figure out if the user has selected valid files for an automated run. Recognizing selected
execution files can simply be done by looking at the file extension.
However determining whether the user selected valid model files that will work with the selected execution
files is somewhat difficult. One possibility would be to use the priority system in the \ac{KIEMConfig} in
order to determine the validity of the combinations. However this would assume that all selected execution files
are known to the plug-in and that the user set priorities for each of them. At this point it is simply assumed
that all selected files that have an extension other than 'execution' are model files. Precautions to avoid
running invalid combinations of model files and execution files are described in Section \ref{section:AutoInput}.

The dialog will only allow the user to proceed if at least one execution file or imported schedule and one
model file is selected. Otherwise an error marker will be displayed in the page header.

\subsection{Property Setting Page}
\label{section:PropertySettingPage}
\begin{figure}[Property Setting Page]
  \centering
  \includegraphics[scale=.4]{PropertySettingPage.png}
  \caption[The Wizard Page for setting up user defined properties.]%
  {The Wizard Page for setting up user defined properties.\protect}
  \label{fig:PropertySettingPage}
\end{figure}
The second wizard page shown in Figure \ref{fig:PropertySettingPage} allows the user to enter some custom
properties for the automated run. Unlike the first page this one doesn't extend a particular wizard page but
rather the generic wizard page that only supplies the header and the button bar at the bottom of the page.

The user can add an arbitrary number of panels for adding new (key, value) pairs through the button at the
top of the page. These values will be added to the list that is passed to all DataComponents before each iteration
and the values can be retrieved by looking for the key.

Since these properties are completely optional there are requirements for finishing this page. Which means that due
to the wizard nature of the dialog the user doesn't even have to look at that page at all but can finish directly
from the file selection page.

As with the file selection page the user input will also be stored as soon as the wizard closes and the properties
restored on the next opening of the wizard. The upside of which is that the user saves considerable work by only
having to enter the properties once. However the downside is that the user might not realize that the properties are
still set if he finishes the wizard directly from the first page. The only way to avoid this situation would be to
force the user to look at the second page. This is however very undesirable since the second page is likely to be used
only by advanced users anyway and the average user wouldn't want to see it.

\subsection{Information Processing}
\label{section:InformationProcessing}
After the user is finished with the wizard all information has to be collected in order to set up the automated run.
This involves the following steps:
\begin{enumerate}
 \item Gather the model files selected on the file selection page.
 \item Get the paths for the execution files. This might involve retrieving the paths from the schedules imported through
the extension point if the user selected any of those.
 \item Get the list of (key, value) pairs that should be added to the automated run from the second wizard page.
 \item Invoke the Automation Manager described in Section \ref{section:AutomatedRun} with the gathered parameters.
\end{enumerate}


\section{Automation Input}
\label{section:AutoInput}
In order to provide the DataComponents with input prior to each part of the automated run new interfaces
had to be created in order to interact with the components. 

\subsection{Automated Component}
An automated component is any DataComponent (see Section \ref{section:IntroDataComponent}) that wants to interact with
the automated execution plug-in. An automated component has to provide the following methods:

\begin{description}
 \item \textbf{String[] getSupportedExtensions()} :

This method is used in order to avoid the automated run encountering errors while trying to
simulate invalid combinations of model files and execution files. As soon as any execution file
is loaded the method will be called on each of the implementing classes. The classes should answer
with a list of file extensions of the model files that they can simulate. Model files that no
component in the currently active execution file can simulate will be skipped.

 \item \textbf{void setParameters(List<KiemProperty> properties) throws KiemInitializationException} : 

This method enables components to receive information prior to each execution
run. The list is implemented as an array of key, value pairs stored inside
KiemProperty objects.
At the every least the list contains the location of the model file and the
index of the currently running iteration.
This allows components to load additional files that are always in the
same path as the execution file and determine which of those to load
based on the iteration index.
The custom properties that the user defined through the wizard for example are
also added here.
If the component encounters an error during at this point because for example a model file
could not be loaded it should respond by throwing the declared Exception.

 \item \textbf{int wantsMoreSteps()} : 

This method is called before the Automation Manager performs the first step.
All components will be asked how many steps they are likely to need for their
execution run. The maximum of these values will be taken and the execution
will perform the requested number of steps. After that the components
will be asked again and so on. The process stops when all components
answer with zero.

 \item \textbf{int wantsMoreRuns()} : 

This method works analog to the wantsMoreSteps() method in the context
of entire execution runs. It is used to determine how many iterations
should be performed with the given combination of execution file and model
file.
\end{description}

\subsection{Automated Producer}
This interface extends the AutomatedComponent interface.
In addition to the inherited methods it provided one additional method.
This method is called after an iteration has finished and asks the components
if they want to publish any information about the results of their execution.
This information is gathered by the plug-in and the accumulated results
are either passed to the calling plug-in or displayed in the
specially designed view (see Section \ref{section:AutoView}).


\section{The Automated Run}
\label{section:AutomatedRun}
The Automation Manager is the key part of the automated execution. It manages the entire control flow
through the automated execution and its public methods are part of the plug-ins \ac{API}. The reason
for this is that the automated run can be initiated without the use of the wizard by any other
plug-in.

The next two sections will give a detailed description of the control flow during the automated execution.
The \ac{API} methods to initiate a new automated execution are located inside the Automation Manager. However
since those methods immediately create a new Automation Job and schedule it right away the first section will
explain the Automation Job. 

The second section will then proceed to describe the entire control flow that is managed by the Automation Manager.

\subsection{Automation Job}
\label{section:AutomationJob}
\index{Automation Job}
The Automation Job is used to run the automated execution in. As a WorkbenchJob it can execute parallel
to the normal operation of the \ac{GUI} without blocking it. It also comes with a progress monitor that
is updated by the Automation Manager through the course of the automated execution. Since an automated run
can take a very long time the user can also tell the job to run in the background while still being able
to get feedback about it through Eclipse's progress view.

Upon creation the Automation Job takes all the parameters necessary for the automated execution. After that
it opens Execution Manager view in order to load all necessary plug-ins before the actual run starts. 
The job then creates a new thread and initiates the automated execution inside the Automation Manager.
At the end it tells the calling thread that it's returning asynchronously in order not to block any callers.

The dialog showing the progress monitor is not only used to get feedback about the progress of the task
it can also be used to cancel the execution prematurely for example if the user realizes that he selected
the wrong files.

\subsection{Automation Manager}
\label{section:AutomationManager}
\index{Automation Manager}
The Automation Manager is responsible for handling the entire control flow during the automated execution.
It takes the execution files, model files and other properties as arguments and organizes the entire
run based on the available information. During the run the Automation Manager also collects the information
that will be displayed as results by the view.


- The control flow:
\begin{itemize}
 \item iterate over all execution files
 \item open execution file
 \item tell view to set up display for the first execution file
 \item iterate over all model files
 \item get first model file from list
 \item ask components how many more runs they need, take maximum and perform runs before asking again
 \item pass model file, execution file and index of iteration
 \item initialize the execution
 \item pass properties to components, at least receive model file and iteration
 \item start worker thread that listens for monitor canceling, step timing out
 \item ask components how many more steps they need, take maximum and perform steps before asking again
 \item perform one step, lock self inside semaphore, stay locked until either worker thread or event listener notifies (step done)
 \item when no component wants more steps pause
 \item gather information from all IAutomatedProducers
 \item tell view to show information for this iteration
 \item stop execution inside the KIEM and perform cleanup
 \item proceed to next iteration
 \item inform monitor of progress
 \item proceed to next model
 \item proceed to next execution
 \item when done inform monitor of done and terminate the job
\end{itemize}

\subsubsection{Modified Error Handler}
One of the problems in the early stages of development was that an automated run wouldn't complete
because of errors in some of the DataComponents.

This was caused by the fact that the Automation Manager operates only indirectly on the execution
that is running inside the Execution Manager and thus will not receive any thrown Exceptions.
Any uncaught Exception in the Execution Manager itself or in any of the DataComponents will
cause the ErrorHandler of the current Eclipse application to be invoked.

The ErrorHandler is responsible for dealing with all errors that may have to be brought
to the users attention. It contains facilities that will allow any plug-in to dispatch an
error with a predefined option of how to handle it. These options include showing the error
in the error log view, opening a dialog or opening a dialog and block the \ac{GUI} until the user
acknowledges the error. During a very long running automated run the last option is very
undesirable as it means that the automation suddenly requires user interaction which of course
defeats the whole point. This is even more annoying for the user since the majority of
Exception during an automated run that cause the \ac{GUI} to block are not critical Exceptions.

For example, one of the DataComponents is analyzing a model file with a given set of trace files.
For some reason one of the trace files is missing which causes an Exception to be thrown inside
the component which throws it to the Execution Manager that called the component. The Execution
Manager doesn't deal with the RuntimeException and throws it up to the Eclipse \ac{GUI} which
responds by invoking the ErrorHandler and blocking the automated execution from continuing. 

If the Exception could have been caught in the Automation Manager it would simply mean that
the iteration with that particular trace file would have to be skipped and the next trace file
should be used. The user then could have received the error at the end of the run or look it up
in the error log.

\listingjava
\showlistingex{code/ErrorHandlerListener.txt}
{Java}
{The interface for listeners on the ErrorHandler.}
{list:ErrorHandlerListener}
{t}
In order to remedy that situation the default ErrorHandler used by Eclipse is replaced with 
a different one. The modified ErrorHandler allows listeners that implement the interface
shown in Listing \ref{list:ErrorHandlerListener} to register to it. Whenever
a plug-in asks the ErrorHandler to deal with an error it will first notify all listeners
of the error that occurred. The listeners then can decide if they want to modify
the status of the given error. The ErrorHandler accumulates all requests from the listeners
and computes the new status. If no listeners are registered or all listeners answered
with ``Don't care'' the error will be handled with its old status. If the listeners
did request a status change all requested status changes will be applied. This
means if one listener asks to only log the error while another listener requests a pop-up
the pop-up dialog is shown and the error is logged.

However there are some errors that will be immediately handled without asking the
listeners first. Those are fatal errors that will likely cause the entire application
to enter an undefined state and where the best course of action usually is to shut down
the application entirely.

In the context of the automated execution the error handler will be asked to only log
the errors while the automated run is in progress.




\section{Automation View}
\label{section:AutoView}
- displays the information in a structured way
- start a new table for each execution file
- one row for each iteration with each model file
 - prerequisite needed here: always the same outputs throughout the entire execution file
- first columns display model file name, iteration index and current status

\subsection{Tool bar}
\label{section:AutoToolbar}
\begin{figure}[Automation View tool bar]
  \centering
  \includegraphics[scale=.8]{AutoToolbar.png}
  \caption[The tool bar in the automation view.]%
  {The tool bar in the automation view.\protect}
  \label{fig:AutoToolbar}
\end{figure}
The tool bar on the Automation View contains several actions for controlling the automated 
execution before, during and after its run (see Figure \ref{fig:AutoToolbar}).
The different controls have the following functionality (left-to-right):
\begin{enumerate}
 \item \textbf{Current Step Field} : During the automated run this field displays the
currently executing step. It is basically the same control as on the tool bar of the 
Execution Manager itself. The control was duplicated in this place in order to avoid
having to switch between the two views. This means that information about the currently
running execution is displayed in the Automation View.
 \item \textbf{Clear} : When the user initiates multiple automated runs after one another
the results are all displayed in the same view. This is the intended behavior as it should 
give the user the ability to compare automated runs with different inputs. However if the 
view gets filled with too many results the user needs an easy way to clear the view which
is realized through this button.
 \item \textbf{Automation Wizard} : The next button is used in order to launch the 
Automation Wizard. The button exists both here and on the Execution Manager's tool bar
in order to allow easy access to the automation.
 \item \textbf{Skip iteration} : During an automated run the user might realize that an
iteration has somehow locked up or isn't aborting because the components keep requesting more
steps. This button allows the user to perform a deferred termination of the current iteration
and proceed to the next index.
 \item \textbf{Skip model file} : This button has the same functionality as the one described
above but it will also skip to the next model file after aborting the current iteration. The reason
for the user wanting to cancel the current model file might be that the components keep requesting
additional runs indefinitely. Another reason would be that the model file keeps producing faulty results
due to the trace files missing and the user doesn't want to wait for it to fail on the remaining files.
 \item \textbf{Skip execution file} : Sometimes the user may even want to skip an entire execution
file if at some point it can be expected that it won't produce any useful results.
 \item \textbf{Cancel automated execution} : The last button in this group is used to initiate a deferred
termination of the entire execution. This has the same functionality as pressing ``Cancel'' inside the 
progress monitor dialog.
 \item \textbf{Export} : The export button is used for opening a dialog to export the currently displayed
results to an external format. This feature is explained in Section \ref{section:AutoExportResults}.
\end{enumerate}

\subsection{Exporting Results}
\label{section:AutoExportResults}

\chapter{Conclusion}
\label{chapter:AutoConclusion}
As stated in Chapter \ref{chapter:AutoTask} the problem consists of four parts:
\begin{enumerate}
 \item Find an easy way for the user to set up an automated run.
 \item Input the information provided by the user into the data components.
 \item Design a control flow for an automated run.
 \item Organize the output of the automated run and display it to the user.
\end{enumerate}
The following sections will summarize the results and provide some ideas for
future work.

\section{Results}
\label{section:AutoResults}
\begin{itemize}
 \item setup using the wizard
 \item input using new interfaces
 \item control flow with job mechanism
 \item output using a view with tables and methods to export
\end{itemize}


\section{Further improvements}
\label{section:AutoImprovements}
\begin{itemize}
 \item allow the user to select which columns to export/display
 \item implement macro step support as soon as KIEM does
\end{itemize}




\appendix

\backmatter
%\include{glossary}
\include{index}
\begin{thebibliography}{cmot-dt}
 \bibitem{cmot-dt}Christian Motika, Semantics and Execution of
 Domain Specific Models, 2009.
 \bibitem{eclipseOverview}Object Technology International, Inc. Eclipse Platform Technical Overview,
2003.
 \bibitem{eclipsePlugins}Eric Clayberg and Dan Rubel, Eclipse Plug-ins. Addison Wesley, 2009.
 \bibitem{evalbench1}Xin Li, The Kiel Esterel Processor: A Multi-Threaded Reactive Processor, July 2007.
 \bibitem{evalbench2}Claus Traulsen and Reinhard von Hanxleden, Reactive Parallel Processing 
for Synchronous Dataflow. In \textit{Proceedings of the 25th  Symposium On Applied Computing (SAC'10), 
Special Track Embedded  Systems: Applications, Solutions, and Techniques}, Sierre, Switzerland, March 2010. To appear.
\end{thebibliography}

%\include{manual}
\chapter*{Appendix}
\section*{Appendix A}
\label{section:AppendixSavedConf}
\listingxml
\showlistingex{code/savedPreferences.txt}
{XML}
{Example for a configuration saved into the Eclipse preference store}
{list:appendixSavedConf}
{t}


\end{document}

%%% Local Variables: 
%%% mode: pdflatex
%%% TeX-master: paper.tex
%%% End: 