
% \part{Erster Teil} % Teile nur bei sehr langen Arbeiten

\chapter{Introduction}
\label{chapter:introduction}
\section{KIELER Framework}
\label{sec:intro/Kieler}
\begin{itemize}
 \item layouter, structure based editing
 \item synccharts
 \item simulater
 \item execution manager
\end{itemize}

\subsection{Model Files}
\label{section:IntroModelFile}
\index{Model file}

\section{The KIELER Execution Manager}
\label{sec:intro/Kiem}
Execution Manager is used in KIELER as a framework to plug-in DataComponents for various tasks. Examples are:
\begin{itemize}
 \item Simulation Engines
 \item Model Visualizers
 \item Environment Visualizers
 \item Validators
 \item User Input Facilities
 \item Trace Recording Facilities 
\end{itemize}

These DataComponents can be executed using a graphical user interface (GUI). 
The scheduling order can also be defined by this GUI as well as other settings like a step/tick duration and properties of DataComponents.
For information about \ac{KIEM} see the wiki\footnote{\url{http://rtsys.informatik.uni-kiel.de/trac/kieler/wiki/Projects/KIEM}}.

\subsection{DataComponents}
\label{section:IntroDataComponent}
\index{DataComponent}

\subsubsection{DataComponentWrappers}
\label{section:IntroDataComponentWrapper}

\subsection{Execution}
\label{section:IntroExecution}
\index{Execution file}
\index{Execution}
\index{Schedule}



\section{Outline of this Document}
\label{sec:intro/Outline}
The first part of this document is about the implemention of the Configuration plugin for the \ac{KIEM}.
The second part discusses the implementation of the Automated Execution plugin for the \ac{KIEM}. Both
parts will have the same structure described below.
Each part starts with a detailed outline of the problems that are to be solved by this thesis.
The next section introduces technologies used to solve the problem.
After that there will be a section about the concepts of how to solve the problem and some
design decisions that were made at a very early stage in the development.
Sections 5 and 6 are about the actual implementation with section 5 outlining the changes
that were made to the \ac{KIEM} plug-in itself and section 6 describing the newly created
\ac{KIEMConfig} plug-in.
The final section of each part will summarize the results of the thesis and outline
a few projects that could be based on it.







%%% Local Variables: 
%%% mode: latex
%%% TeX-master: "paper"
%%% End: 

