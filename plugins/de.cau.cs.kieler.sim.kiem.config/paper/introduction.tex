\chapter{Introduction}
\label{chapter:introduction}
The purpose of this thesis consists of two parts. The first part is to find an easier way and
more flexible way to deal with execution files in the \ac{KIELER} Execution Manager. The second part
is to find an easy way to automatically do long execution runs inside the \ac{KIEM}.

\section{KIELER Framework}
\label{sec:introKieler}
\index{KIELER}
Since the project is part of the \ac{KIELER}\footnote{\url{http://www.informatik.uni-kiel.de/rtsys/kieler/}}
framework a short introduction is in order.

\ac{KIELER} is an open-source project for model design, simulation and analysis. 

It contains facilities to edit different types of graphical models (e.g. synccharts) and has several
tools to make the editing easier. The \ac{KSBasE}\footnote{\url{http://rtsys.informatik.uni-kiel.de/trac/kieler/wiki/Projects/KSBasE}} 
project for example contains several features where the user can edit successor states to already existing states in the diagram or
adding new inner states without having to perform all the necessary actions himself. Another 
project\footnote{\ac{KIML} - \url{http://rtsys.informatik.uni-kiel.de/trac/kieler/wiki/Projects/KIML}}
automatically layouts any diagram with a variety of algorithms in order to improve readability.

The \ac{KIELER} Execution Manager described in the next section is used for simulating the
model files created by \ac{KIELER}.

\subsection{Model Files}
\label{section:IntroModelFile}
\index{Model file}
A model file can be any file that contains the structure and behavior of a semantic entity.

\section{The KIELER Execution Manager}
\label{sec:introKiem}
\index{KIEM}
Execution Manager is used in KIELER as a framework to plug-in DataComponents for various tasks. Examples are:
\begin{itemize}
 \item Simulation Engines
 \item Model Visualizers
 \item Environment Visualizers
 \item Validators
 \item User Input Facilities
 \item Trace Recording Facilities 
\end{itemize}

These DataComponents can be executed using a graphical user interface (GUI). 
The scheduling order can also be defined by this GUI as well as other settings like a step/tick duration and properties of DataComponents.
For information about \ac{KIEM} see the wiki\footnote{\url{http://rtsys.informatik.uni-kiel.de/trac/kieler/wiki/Projects/KIEM}}.

\subsection{DataComponents}
\label{section:IntroDataComponent}
\index{DataComponent}
A DataComponent is an entity that has a specific task during a the course of an execution. The
DataComponents are scheduled in a specific order and can either receive or produce information or both. 
During each step of the execution each DataComponent is asked to perform their computations for the step.

\subsubsection{DataComponentWrappers}
\label{section:IntroDataComponentWrapper}
\index{DataComponentWrapper}
A DataComponentWrapper is an object that wrapps around a DataComponent that is contained in the schedule
of an execution file.


\subsection{Execution}
\label{section:IntroExecution}
\index{Execution file}
\index{Execution}
\index{Schedule}
The files used by the Execution Manager are called \textit{execution files}. These files contain 
the list of DataComponents with all internal data and the specific order. That list is called a
\textit{schedule}. The schedule can be used to start an \textit{execution} in the Execution Manager.
An execution consists of a initialization, stepping and wrap-up.


\section{Outline of this Document}
\label{sec:introOutline}
The first part of this document is about the implemention of the Configuration plugin for the \ac{KIEM}.
The second part discusses the implementation of the Automated Execution plugin for the \ac{KIEM}. Both
parts will have the same structure described below.

Each part starts with a brief introduction into the technologies used to solve the problem.

The next chapter contains a detailed outline of the problems that are to be solved by this thesis.

The following chapter describes the concepts of how to solve the problem and some
design decisions that were made at a very early stage in the development.

Sections 5 and 6 are about the actual implementation with section 5 outlining the changes
that were made to the \ac{KIEM} plug-in itself and section 6 describing the newly created plug-ins.

The final section of each part will summarize the results of the thesis and outline
a few projects that could be based on it.


%%% Local Variables: 
%%% mode: latex
%%% TeX-master: "paper"
%%% End: 

