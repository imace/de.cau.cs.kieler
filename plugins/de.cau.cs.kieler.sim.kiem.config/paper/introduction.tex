\chapter{Introduction}
\label{chapter:introduction}
The purpose of this thesis consists of two parts. The first part is to find an easier way and
more flexible way to deal with execution files in the \ac{KIELER} Execution Manager (see Section
\ref{section:introKiem}). The second part
is to find an easy way to automatically do long execution runs inside the \ac{KIEM}.

\section{KIELER Framework}
\label{section:introKieler}
\index{KIELER}
Since the project is part of the \ac{KIELER}\footnote{\url{http://www.informatik.uni-kiel.de/rtsys/kieler/}}
framework a short introduction is in order.

\ac{KIELER} is an open-source project for model design, simulation and analysis. It is developed 
by the Real-Time and Embedded Systems Group\footnote{\url{http://www.informatik.uni-kiel.de/rtsys/}}
of the Department of Computer Science\footnote{\url{http://www.informatik.uni-kiel.de}}
of the \ac{CAU}\footnote{\url{http://www.uni-kiel.de}}.

It contains facilities to edit different types of graphical models (e.g. synccharts) and has several
tools to make the editing easier. The \ac{KSBasE}\footnote{\url{http://rtsys.informatik.uni-kiel.de/trac/kieler/wiki/Projects/KSBasE}} 
project for example contains several features where the user can edit successor states to already existing states in the diagram or
adding new inner states without having to perform all the necessary actions himself. Another 
project\footnote{\url{http://rtsys.informatik.uni-kiel.de/trac/kieler/wiki/Projects/KIML}}
automatically layouts any diagram with a variety of algorithms in order to improve readability.

The \ac{KIELER} Execution Manager described in the next section is used for simulating the
model files created by \ac{KIELER}.


\section{Outline of this Document}
\label{section:introOutline}
The first part of this document is about the implemention of the \ac{KIEMConfig} plug-in.

It starts with an introduction into the technologies that were used to solve the problem as well
as an overview of similar projects. The part continues with a detailed description of the problem
followed by a chapter about a conceptual solution to those problems. The next chapter is about the
modifications that had to be made to the existing Execution Manager. In the following chapter a detailed
description of the implementation of the new features will be given. The last chapter will summarize
the results of this thesis and outline a few projects that could follow up on it.

The second part discusses the implementation of the \ac{KIEMAuto} plug-in. It follows the same structure
as the first part.


%%% Local Variables: 
%%% mode: latex
%%% TeX-master: "paper"
%%% End: 

