\chapter{Problem Statement}
\label{chapter:AutoTask}
The objective of this project is to find a way to automate the execution
runs of the Execution Manager as described by Christian Motika in his
diploma-thesis\cite{cmot-dt}.

Currently the Execution manager works in a way that the user manually sets up
a new execution run or loads a saved execution file. The DataComponents then have 
to gather all information they need themselves like model files, trace files and so on.
Since there is no generic way to do that, this information is either hard-coded into
the components or entered manually through the properties.
After that the user has to manually control the execution. The execution runs until
the user or a component stops it. The user then has to manually set up another
execution run, possibly even rewriting his components if the model files
and trace files are hard-coded or manually change the properties.
This is very unsatisfactory if you have a large number of model files that
should be tested with a one or more execution files and possibly hundreds of trace files
(see Figure [TODO]).
% ask ctr for screen shot of directory perhaps
Performing runs like that manually is completely out of the question as even
with the automation in place it would take several hours.

The task resulting from this problem can be broken down in 4 parts which are explained in detail in
the following sections:
\begin{enumerate}
 \item The setup of an automated run by the user.
 \item The input of all the necessary information.
 \item The control flow of the automated run.
 \item The gathering of information after the run has finished and the display 
of that information.
\end{enumerate}

\section{Setting up a Run}
\label{section:AutoTaskSetup}
The first objective is to find an easy way for the user to efficiently set up an
automated run. This involves selecting the model files and execution files
needed for the automation as well as entering initial properties.

\section{Input for the Automation}
\label{section:AutoTaskInput}
The second objective is to enable the components to receive inputs.
Each component should receive all information it needs prior to each execution
run in order to make the components more dynamic. This mechanism would ensure
that components can be written in a more generic way than is currently possible.
We will have to define an \ac{API} for this information passing process as well as
an \ac{API} to trigger an automated execution from other plug-ins.


\section{Automate the execution}
\label{section:AutoTaskExecution}
The third objective is to automate the control flow of the execution itself. This would
involve the following:
\begin{enumerate}
  \item Loading the desired execution files, model files and trace files.
  \item Determining how many steps should be performed and running the execution up the desired step.
  \item Gathering the information produced by the components.
  \item Properly shut down the execution so that a new one can be started.
\end{enumerate}


\section{Output execution results}
\label{section:AutoTaskOutput}
The last objective is to display the information in a meaningful way.
This should involve at least two methods of output:
\begin{enumerate}
  \item A formatted string possibly in an XML fashion that can be parsed and
  used by other plug-ins for automated analysis.
  \item Some graphic component that will display the information in a way that is
  easy to read for most users.
\end{enumerate}

