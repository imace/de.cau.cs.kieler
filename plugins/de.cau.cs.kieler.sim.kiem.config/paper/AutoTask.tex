\chapter{Problem Statement}
\label{chapter:AutoTask}

In order to do validations or recording automatically in a batch-like mode, 
the Execution Manager needs to be extended by an Eclipse Plug-in that some kind of 
remote controls it (API) and has/supports for example: 
\begin{itemize}
 \item An own eclipse view (GUI) 
 \item Loading and saving batch files 
 \item Some way to specify batch jobs (e.g., a special notation language) 
 \item Possibly a way to do all this from the command line
\end{itemize}


Currently the KIEM works in a way that you select an execution file and then
press play. The components then have to gather all information they need
themselves like model files, trace files and so on. The execution runs until
the user or a component stops it. The user then has to manually set up another
execution run, possibly even rewriting his components if the model files
and trace files are hard coded.
This is very unsatisfactory if you have a large number of model files that
should be tested with a one or more execution files and several trace files.
% ask ctr for screenshot of directory perhaps

The problem can be broken down in 4 parts which are explained in detail in
the following sections:
\begin{enumerate}
 \item The setup of an automated run by the user.
 \item The input of all the necessary information.
 \item The control flow of the automated run.
 \item The gathering of information after the run has finished and the display 
of that information.
\end{enumerate}


\section{Setting up a Run}
\label{section:AutoTaskSetup}
The first objective is to find an easy way for the user to efficiently set up an
automated run. This involves selecting the model files and execution files
needed for the automation as well as entering initial properties.

\section{Input for the Automation}
\label{section:AutoTaskInput}
The second objective is to enable the components to receive inputs.
Each component should receive all information it needs prior to each execution
run in order to make the components more dynamic. This mechanism would ensure
that components can be written in a more generic way than is currently possible.
We will have to define an API for this information passing process as well as
an API to trigger an automated execution from other plug-ins.


\section{Automate the execution}
\label{section:AutoTaskExecution}
The third objective is to automate the process of execution itself. This would
involve the following:
\begin{enumerate}
  \item loading an execution file
  \item stepping the execution to the required step
  \item gather the information produced by the components
  \item properly shut down the execution so that a new one can be started
\end{enumerate}


\section{Output execution results}
\label{section:AutoTaskOutput}
The last objective is to display the information in a meaningful way.
This should involve at least two methods of output:
\begin{enumerate}
  \item A formatted string possibly in an XML fashion that can be parsed and
  used by other plug-ins for automatic analysis.
  \item Some graphic component that will display the information in way that is
  easy to read for most users.
\end{enumerate}

