\chapter{Conclusion}
\label{chapter:ConfConclusion}
As stated in Chapter \ref{chapter:ConfTask} the problem consists of two parts:
\begin{enumerate}
 \item Find a way to add configurations to the existing execution files. Additionally find
a way to allow the user the set up default preferences.
 \item Make it easier to load previously saved schedules without having to locate
the execution file in the workspace.
\end{enumerate}
The following sections will summarize the results and provide some ideas for
future work.

\section{Results}
\begin{itemize}
 \item problems solved
\end{itemize}

\section{Further Improvements}
Although all initial goals of this thesis were more than met there are still
some features that could not be added due to the lack of time.
\subsection*{Eclipse Runtime Mechanism}
The Eclipse framework provides a very comprehensive system to run different
modules. This is used to execute Java programs or to start a new Eclipse application
but there are a variety of other applications as well.

The entire Execution Manager could be refactored into using that runtime mechanism instead
of setting up a run through the now present \ac{KIEM} view. This means that the table
that shows the DataComponents has to be moved to a new runtime page. 

The controls for pausing, resuming, stopping and stepping through the execution have to
be moved somewhere else as well, possibly the DataTable.

\subsection*{Improve Storage Options}
Currently DataComponents as well as the preference mechanism only allow the use of
Strings to store the preferences. This means that all primitive data types can more
or less be stored by conversion to a String. However more complex objects can't be stored
without serializing them into a String and parsing them again on load.

A future project could try to find a way to overcome that limitation by allowing
objects that implement the Serializable interface to be stored as well.