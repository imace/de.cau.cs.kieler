\chapter{Conclusion}
\label{chapter:ConfConclusion}
As stated in Chapter \ref{chapter:ConfTask} the problem consists of two parts:
\begin{enumerate}
 \item Find a way to add configurations to the existing execution files. Additionally find
a way to allow the user the set up default preferences.
 \item Make it easier to load previously saved schedules without having to locate
the execution file in the workspace.
\end{enumerate}
The following sections will summarize the results and provide some ideas for
future work.

\section{Results}
The first task was to implement a way for execution files to carry configuration
properties. This was solved by creating a new type of DataComponent that stores all
properties and is accessed by the \ac{KIEMConfig} when values for the properties are needed.
The newly created component is automatically saved with any execution file that wants to
use the new feature.

Part of the first task involved enabling the user to manage a default configuration
for the different properties. This was realized through the creation of a set
of preference pages. These pages allow the user to manage the properties and even
override the ones stored in the execution file.

The last objective concerned the actual loading of the execution files. A way had to be
found in order to make loading the files easier and provide filtered perspectives on all known
files. This was accomplished by providing the user with two ComboBoxes that contained a
subset of all known execution files. One ComboBox is used to load the most recently used files
while the user contains files that probably work for the currently active editor.

\section{Future Work}
Although all initial goals of this thesis were met there are still some ideas left which
could solve as a basis for further study.
\subsection{Eclipse Run Configurations}
The Eclipse framework provides a very comprehensive system to run different
modules. This is used to execute Java programs or to start a new Eclipse application
but there are a variety of other applications as well.

The entire Execution Manager could be refactored into using that runtime mechanism instead
of setting up a run through the now present \ac{KIEM} view. This means that the table
that shows the DataComponents has to be moved to a new runtime page. 

The controls for pausing, resuming, stopping and stepping through the execution have to
be moved somewhere else as well, possibly the DataTable.

\subsection{Improve Storage Options}
Currently DataComponents as well as the preference mechanism only allow the use of
Strings to store the preferences. This means that all primitive data types can more
or less be stored by conversion to a String. However more complex objects can't be stored
without serializing them into a String and parsing them again on load.

A future project could try to find a way to overcome that limitation by allowing
objects that implement the \textit{Serialization} interface to be stored as well.

\subsection{Use Advanced KiemPropertyTypes}
Currently the \ac{KIEMConfig} can only handle KiemProperties of the basic type that
contains a String value. However there is a whole range of different KiemPropertyTypes
such as integer, files or choice. An enhancement to the \ac{KIEMConfig} would be to
utilize the full potential of these types.

This would mean rewriting much of the already existing code and completely refining
the page where the user can define their properties.

\section{Summary}
All in all the initial problem of providing a means to storing additional configuration information
in an execution file as well as providing a mechanism for providing a default configuration was solved. 
In addition to that the second objective which involved finding a way to make it easier to load previously
saved schedules was also completed.

However there is still a lot of basic for further work on the concepts of this project and on the
concept of the Execution Manager in general.