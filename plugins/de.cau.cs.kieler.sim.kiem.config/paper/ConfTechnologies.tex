\chapter{Used technologies}
\label{chapter:ConfTechnology}
- before explaining solutions

- short explanation about the technology in question

- small sketch to get an idea of the context

- full explanation goes beyond the scope of this work

\section{Eclipse}
\index{Eclipse}
- KIELER part of Eclipse framework

- Java IDE but also lots of other languages (C++, Latex, ...)

- can build other IDEs with it

- citation: IDE an for anything, and nothing in particular \cite{eclipseOverview}

\subsection{Plug-ins}
\index{Plug-in}
- \ac{KIEM} itself is a plug-in, this work plug-in dependent on \ac{KIEM}

- different components of an \ac{IDE} 

- can operate by themselves or depending on other plug-ins

- in addition to \ac{API} of each plug-in: extension point mechanism

- extension points to interact with other plug-ins

- extension point mechanism used by runtime environment to check which plug-ins to load

- plug-ins can be used to plug into the standard Eclipse feature
\subsubsection{Extension point mechanism}
\label{section:ConfTechnologyExtension}
\index{Extension point}


\subsection{Preference Pages}
\index{Preference Page}
- One of the plug-ins used in this work in that fashion is the \emph{org.eclipse.ui.preferencePage}
plug-in. It is used to create new preference pages at a specific location inside the
normal tree of preference pages accessible through Window>Preferences.
The programmer only has to take care of the contents of the actual page and not worry
about additional buttons or integrating it into the PreferenceDialog.

\subsubsection{Preference Store}
- Closely coupled with the preference pages is the Eclipse preference store. It is
basically a text file for each plug-in where the plug-in can deposit simple Strings
under a given key to ensure that information is kept between execution runs.

For additional information about eclipse see the official Eclipse website\footnote{www.eclipse.org} or literature \cite{eclipsePlugins}