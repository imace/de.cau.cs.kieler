\chapter{The Automated Executions Plug-in}
- new plugin
- handles the setup, control flow and display of automated execution
- consists of 3 parts explained in detail below
- wizard, manager, view
\section{The wizard}
\subsection{File selection page}
- wizard is used to set up the execution run
- extends ResourceImportWizard for displaying a folder/file structure for selecting files from
- easily usuable, select whole folders, filter file types
- can be given an initial selection, on close will save the selection, store it in preference store
and restore it on load
- additional dialog for selecting execution files that are not in the workspace but imported
- for simplicity assume that files ending .execution are execution files and all other selected
ones are model files, wizard can not check if valid since formats are not known
- only allow user to proceed if at least one execution and one model file is selected
\subsection{Property Setting page}
- set up the additional arguments passed to the execution
- simple adding and removing of key, value pairs
- same as file page, on close results are saved to preference store and restored for initial
selection on next open
\subsection{Processing the information}
- gather execution files and model files from file selection page
- gather properties from property page
- store information for next open
- invoke the execution manager

\section{The Execution manager and job}
- handles control flow during the automation
- takes information from either call through the API or wizard

\subsection{Execution Manager}
- handles the overall control flow
- takes the execution files, model files and properties as argument
- if progress monitor is registered it is informed about the progress of the evaluation
- The control flow:
\begin{itemize}
 \item iterate over all execution files
 \item open execution file
 \item tell view to set up display for the first execution file
 \item iterate over all model files
 \item get first model file from list
 \item ask components how many more runs they need, take maximum and perform runs before asking again
 \item pass model file, execution file and index of iteration
 \item initialize the execution
 \item pass properties to components, at least receive model file and iteration
 \item start worker thread that listens for monitor canceling, step timing out
 \item ask components how many more steps they need, take maximum and perform steps before asking again
 \item perform one step, lock self inside semaphore, stay locked until either worker thread or event listener notifies (step done)
 \item when no component wants more steps pause
 \item gather information from all IAutomatedProducers
 \item tell view to show information for this iteration
 \item stop execution inside the KIEM and perform cleanup
 \item proceed to next iteration
 \item inform monitor of progress
 \item proceed to next model
 \item proceed to next execution
 \item when done inform monitor of done and terminate the job
\end{itemize}



\subsection{Execution Job}
- workbenchjob with progressmonitor
- used to display the progress in the progress view and a dialog with progress bar
- long running task, doesn't want to block the rest of the workbench 
- triggers execution in the manager

\section{The View}
- displays the information in a structured way
- start a new table for each execution file
- one row for each iteration with each model file
 - prerequisite needed here: always the same outputs throughout the entire execution file
- first columns display model file name, iteration index and current status

\subsection{Toolbar}
- button to start the wizard
- button for clearing the view
- text field showing the step that was just processed
