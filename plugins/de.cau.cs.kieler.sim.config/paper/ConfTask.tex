\chapter{Problem Statement}
\label{chapter:ConfTask}
The objective of this project is to improve the configurability of the
\ac{KIEM} as outlined in the Diploma-thesis by Christian Motika \cite{cmot-dt}:
\begin{quote}
 \ac{KIEM} currently does not have a preference page to save
additional settings like DataComponent timeouts. Also execution schedulings
might be similar for a common diagram type.

It may improve the usability further to allow the user to customize execution
schedulings for specific diagram types. An interface for these kind of settings
could be realized as an Eclipse preference page.
\end{quote}
This will be explained in more detail in the following sections.
\section{Configurations}
Currently every property in the \ac{KIEM} has a hard coded default value. There is a text box
for setting the aimed step duration for the currently loaded execution file but that value
is lost once a new execution is loaded.
To solve this problem an extension to the \ac{KIEM} should attempt to provide the following:
\begin{enumerate}
 \item Find a way that execution files can store values like the aimed step duration and the timeout.
This mechanism should be implemented in a way that ensures that old files can be upgraded and new
files are still valid in instances of the \ac{KIEM} that don't use the configuration plug-in.
 \item Find a way to load the configurations into the different parts of the \ac{KIEM} as soon as an
execution file is loaded from the file system.
 \item Ensure that the user can edit all properties and maybe even create his own custom properties.
This should be implemented in a way that doesn't clutter up the current user interface too much.
\end{enumerate}

%\begin{itemize}
% \item different configurable elements in KIEM (Timeout, Aimed step duration)
% \item currently hard coded
% \item way that an .execution file can carry their own configuration without making old files invalid
% \item load/save configurations
%\end{itemize}

\subsection{Default Configuration}
\index{Default Configuration}
The different properties stored in each execution file might sometimes not suit the users current needs
and he might want to use a default value for some properties without having to manually set them
in each new configuration.
The solution could be implemented in a preference page that contains the following:
\begin{enumerate}
 \item a way to set the default properties for all \ac{KIEM} properties and possibly for user defined properties as well.
 \item a way to set which of these properties should override the value stored in the execution file and which
should only be used if the execution file doesn't contain one.
\end{enumerate}

%\begin{itemize}
% \item don't want to set properties on every current configurations
% \item default values configurable through a preference page
% \item override values used in a configuration with defaults
%\end{itemize}

\section{Easier configuration loading}
The last objective of this thesis is to make it easier to load execution files.
Currently all execution files are stored in the workspace at a place of the users choice. In a very
large workspace it can be very hard to find the execution file that you need for your current
simulation. The list of recently used documents that Eclipse provides is of little use since all
opened documents are placed there not just execution files.
This problem leads to the in the following tasks:
\begin{enumerate}
 \item Finding a way to track recently used execution files and make it easier for the user to
load them without having to locate them inside his workspace.
 \item In addition to tracking recently used execution files the user might want to have a way to have a list
of execution files that work for the currently active editor. This list should be sorted with the most likely
candidates at the top to allow less experienced users to select an execution file that will most likely work.
\end{enumerate}


%\begin{itemize}
% \item track recently used schedules
% \item load most recently used schedule on startup
% \item display filtered list based on currently active editors
%\end{itemize}