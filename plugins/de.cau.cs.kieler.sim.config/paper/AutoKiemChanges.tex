\chapter{Code changes in KIEMPlugin}
\label{chapter:AutoKiemChanges}
\begin{itemize}
 \item interfaces and API changes to allow access to execution
 \item changes to load values rather than hard coded defaults
\end{itemize}

\section{Schema files and Interfaces}
\begin{itemize}
 \item event listener mentioned in part I, for listening to KIEM execution
 events
 \item interfaces added to KIEM itself to avoid components breaking when
 automated plug-in is not loaded
\end{itemize}

\section{Automated Component}
An automated component is any DataComponent that wants to interact with
the automated execution plug-in. As seen in the diagram automated components
have to implement three methods:

\subsection{provideProperties()}
This method enables components to receive information prior to each execution
run. The list is implemented as an array of key, value pairs stored inside
KiemProperty objects.
At the every least the list contains the location of the model file and the
index of the currently running iteration (that is how many time the current
model has already been executed with the current execution file).
This allows components to load additional files that are always in the
same path as the execution file and determine which of those to load
based on the iteration index.

\subsection{wantAnotherStep()}
This method is called after every step of the execution manager.
All components are asked if they want to perform another step and if
one components answers with TRUE the execution manager will perform another
step.
This makes it unnecessary to know the exact number of steps needed for
the execution before the execution starts.

\subsection{wantsAnotherRun()}
This method is the equivalent for the wantAnotherStep() method in the context
of entire execution runs.
After no component wants to perform another step all components are asked
if they want to perform another run. If one component answers with TRUE
the entire simulation is stopped and then started again with the same
model file and execution file but an incremented iteration index.
This can for example be used to execute the same model with a different
set of trace files by just naming the trace files the same way as the
model files with a number at the end.

\section{Automated Producer}
This interface extends the AutomatedComponent interface.
In addition to the inherited methods it provided one additional method.
This method is called after an iteration has finished and asks the components
if they want to publish any information about the results of their execution.
This information is gathered by the plug-in and the accumulated results
are either passed to the calling plug-in or displayed in the
specially designed view (see chapter about the view).
