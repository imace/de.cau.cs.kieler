\chapter{Bedienungsanleitung}\index{Bedienungsanleitung}
Um mit der Uhr arbeiten zu k"onnen muss man als erstes eine Batterie
einlegen. Die Uhr wechselt dann sofort in den Macrozustand \zst{alive}, und
dort in den Unterzustand \zst{displays}. Von nun an werden Sekunden Minuten
und Stunden auf dem LCD-Display angezeigt. 


\begin{description}
\item[Dr"ucken von \kn{Knopf d}:]
M"ochte der Benutzer der Uhr sich nun  das datum ausgeben lassen, dann dr"uckt
er den \kn{Knopf d}. Die Uhr wechselt nun in den Zustand \zst{date} und das 
Datum
erscheint auf der LCD-Anzeige. 

\item[Dr"ucken von \kn{Knopf a}:]
Um in das uhr-Men"u zu gelangen muss man nur den \kn{Knopf a} dr"ucken. 
Folgende mehrfachbenutzung ist m"oglich:
\begin{description}
\item[Bei einfachem Dr"ucken von\kn{ Knopf a}]
gelangt man in das Men"u von Alarm1. Mit Hilfe des \kn{Knopfes d} wird der 
Alarm1
in den Zustand \zst{on} oder durch erneutes Dr"ucken in den Zustand  \zst{off}
geschaltet. 
Befindet sich der Alarm1 im Zustand \zst{on}, dann erscheint eine kleine
Glocke mit einer 1  oben links in der Ecke der Uhr. 
Befindet sich nun der Alarm1 im Zustand \zst{off}, dann erlischt die kleine
Glocke wieder. 

Um den Alarm zu stellen muss der Benutzer den \kn{Knopf c} dr"ucken. Nun kann 
man mit dem \kn{Knopf d} die Stunden erh"ohen. Ist die gew"unschte einstellung
vorgenommen, dann gelangt man "uber das dr"ucken von \kn{Knopf c} zum Stellen 
der Zehnerstelle der Minuten. Nach fertigem einstellen gelangt man wieder "uber
einen Druck auf \kn{Knopf c} zum Stellen der Minuten. Sind nun alle 
Einstellungen vorgenommen worden, dann gelangt man mit Hilfe des 
\kn{Knopfes c} zur"uch ins Men"u von Alarm1. 
Um wieder in den \zst{time}-Zustand zu gelangen, in dem die Uhrzeit
angezeigt wird muss man noch viermal den Knopf a dr"ucken. 

\item[Bei zweifachem Dr"ucken von \kn{Knopf a}]
gelangt man in das Men"u von Alarm2. Das Men"u von Alarm2 ist genauso
aufgebaut, wie das Men"u von Alarm1 (s.o.). Mit der ausnahme, dass man beim
verlassen des Zustandes durch dreifaches Dr"ucken in den Zustand \zst{time}
gelangt. 
 
\item[Bei dreifachem Dr"ucken von \kn{Knopf a}]
gelangt man in das Men"u des Zustands \zst{Chime}. Mit Hilfe des \kn{Knopfes d}
kann man nun einstellen, dass die Uhr alle volle Stunde einmal  alarm schlagen
soll. Durch erneutes Dr"ucken des Knopfes d wird diese Funktion wieder
ausgeschaltet. Ist der Zustand \zst{Chime} aktiv, dann erscheint auf dem
Uhr-\eng{display} eine kleine gelbe Glocke. Ist der Zustand inaktiv, dann 
erlischt sie wieder.  
Um wieder in den \zst{time}-Zustand zu gelangen muss man noch zweimal den 
Knopf a dr"ucken. 

\item[Bei vierfachem Dr"ucken von Knopf a] gelangt man in das Stopuhr-Men"u.
  Die LCD-Anzeige erweitert sich automatisch um die M"oglichkeit nun auch \cs
  anzeigen zu k"onnen. Durch das Dr"ucken von \kn{Knopf c} startet man die 
Stopuhr. Um nun die Zeit zu stoppen gen"ugt ein erneutes Dr"ucken des 
\kn{Knopfes c}.  M"ochte man die Stopuhr wieder auf Null setzen, dann muss 
man den \kn{Knopf d}  dr"ucken.  In den Zustand \zst{time} gelangt man durch 
einfaches Dr"ucken  von \kn{Knopf a}.
\end{description}

\item[Dr"ucken von \kn{Knopf b}:]
Die LCD-Anzeigen Beleuchtung der Uhr erscheint durch das Dr"ucken von \kn{Knopf
b}. Wird nun nocheinmal dieser Knopf gedr"uckt, dann erlischt das Licht
wieder. 
\item[Batterie:]
Ist die Batterie der Uhr schwach (Das Signal \signal{Batt\_weakens} wird
emittiert), dann beginnt die Uhr zu blinken. Wird die Batterie
herausgenommen oder erlischt, dann bleibt die Uhr stehen. 
\end{description}

%%% Local Variables: 
%%% mode: latex
%%% TeX-master: "paper"
%%% End: 
